\documentclass{beamer}

\usepackage[utf8]{inputenc}

\beamertemplatenavigationsymbolsempty
\setbeamertemplate{footline}{
	\raisebox{0.5em}{\makebox[\paperwidth-0.5em][r]{
		\color{gray}\scriptsize\insertframenumber}}}

\usepackage{mathtools}
\usepackage{tikz}
\usetikzlibrary{cd, shapes, tikzmark}
\def\pathToRoot{../../}\usepackage{nag}
\makeatletter
\@ifclassloaded{beamer}{}
{\usepackage[small,compact]{titlesec}}
\makeatother
\usepackage[utf8]{inputenc}
\usepackage[T1]{fontenc}
\usepackage{lmodern}
\usepackage{color}
\usepackage{parskip}
\usepackage{needspace}
\usepackage{microtype}
\usepackage{mathtools}
\usepackage{xifthen}
\usepackage{xpatch}
\usepackage{enumitem}
\usepackage{mdwlist}
\usepackage{bussproofs}
\EnableBpAbbreviations
\usepackage{tabu}
\usepackage{amssymb}
\usepackage{amsmath}
\usepackage{amsthm}
%grober hack, der den groben hack von parskip bei den amsthm sachen korrigiert
\begingroup
    \makeatletter
       \@for\theoremstyle:=definition,remark,plain\do{%
            \expandafter\g@addto@macro\csname th@\theoremstyle\endcsname{%
                        \addtolength\thm@preskip\parskip
             }%
        }
\endgroup
\usepackage[UKenglish]{babel}
\usepackage{xparse}
\usepackage{adjustbox}
\usepackage{geometry}
\usepackage{booktabs}
\usepackage{multicol}
\usepackage{soul}
\usepackage{calc}
\usepackage{textcase}
\usepackage{stmaryrd}
\usepackage{marvosym}
\usepackage{wasysym}
\usepackage{pifont}
\newcommand{\cmark}{\ding{51}}
\newcommand{\xmark}{\ding{55}}
\usepackage{tikz}
\usetikzlibrary{trees, backgrounds, shapes, chains, decorations.text, decorations.pathreplacing, circuits.logic.IEC, patterns, matrix}
\usepackage{tikz-qtree}
\usepackage{tikzsymbols}
\usepackage{fancyvrb}
\usepackage{fancyhdr}
\usepackage{verbatim}
\usepackage[framemethod=tikz]{mdframed}
\usepackage{lastpage}
\usepackage{pgfpages}
\usepackage{csquotes}
\usepackage{longtable}
\usepackage{ragged2e}
%\usepackage{stackengine}
\usepackage{censor}
\usepackage{expl3}
\usepackage{multirow}
\usepackage{hyperref}
\usepackage{environ}


% Package for Cateogry diagrams:

\usepackage{tikz-cd}



\ifcsdef{labelenumi}{
\renewcommand{\labelenumi}{(\alph{enumi})}
\renewcommand{\labelenumii}{(\roman{enumii})}
}{}

\input{\pathToRoot headers/definitions}



\tikzset{
    normal/.style={draw, semithick},
    n/.style={style=normal, circle, inner sep=1mm, minimum size=8mm},
    l/.style={style=normal, rounded corners=1mm, inner sep=1mm, minimum size=6mm},
    e/.style={style=normal, shorten >=1mm, shorten <=1mm, ->, >=stealth},
    syntax/.style={style=normal, ellipse, minimum height=6mm, minimum width=8mm}, % nodes in syntax trees
    inner/.style={style=normal, minimum size=4mm}, % inner leaves or root in normal trees
    leaf/.style={style=normal, circle, minimum size=4mm}, % leaves in normal trees
    te/.style={style=normal}, % edges in a tree
    be/.style={style=e, dashed} % binding edge
}

\newcommand{\syntaxtree}[1]{ % DEPRECATED - use tikzsyntaxtree
    \begin{tikzpicture}[baseline=(current bounding box.north)]
        \tikzset{grow=down}
        \tikzset{every node/.style={syntax}}
        \tikzset{edge from parent/.style=
            {te,
                edge from parent path={(\tikzparentnode) -- (\tikzchildnode)}}}
        \Tree #1
    \end{tikzpicture}
}

\newenvironment{tikzsyntaxtree}[1][]{
    \begin{tikzpicture}[baseline=(current bounding box.north), #1]
    \tikzset{grow=down}
    \tikzset{every tree node/.style={syntax}}
    \tikzset{edge from parent/.style={te, edge from parent path={(\tikzparentnode) -- (\tikzchildnode)}}}
}{
    \end{tikzpicture}
}


\newcommand{\DisplayScaledProof}{\maxsizebox{\linewidth}{!}{\DisplayProof}}
\newcommand{\DisplayTopProof}{\adjustbox{valign=t}{\DisplayProof}}
\newcommand{\DisplayScaledTopProof}{\adjustbox{valign=t}{\maxsizebox{\linewidth}{!}{\DisplayProof}}}


\newcolumntype{P}[1]{>{\RaggedRight\hspace{0pt}}p{#1}}

\newenvironment{prooftable}
{
    \begin{longtable}{>{\footnotesize}p{0.33\textwidth}>{\footnotesize}p{0.33\textwidth}|>{\footnotesize}P{0.15\textwidth}}
    \normalsize Textbeweis & \normalsize Erklärungen & \normalsize Schlussregel\\\hline
    \endhead
}
{
    \end{longtable}
}


\theoremstyle{definition}
\newtheorem*{definition*}{Definition} % Definition ohne Nummer
\newtheorem*{inferenceRule*}{Schlussregel}

%\input{commands}


\title{Special Constructions in Categories}
\subtitle{Examples for limits and colimits}
\author{Maximilian Wuttke}
\date{June 7, 2017}

\begin{document}

\frame{\titlepage}

\begin{frame}
  \frametitle{Table of contents}
    \begin{table}
    \begin{tabular}{l | l}
      Limit & Colimit \\ \hline
      Product & Coproduct \\
      Equalizer & Coequalizer \\
      Pullback & Pushout
    \end{tabular}
    \end{table}
\end{frame}

\begin{frame}
  \frametitle{Product}

  \begin{tabular}{l p{7cm}}

    \xymatrix{
      X & Y
    }

    & Let $\cat{A}$ be a category and $X, Y \in \cat{A}$.
    A \demph{product} of $X$ and $Y$ consists of an object $P$ and maps \\

    \xymatrix{
              &P \ar[ld]_{p_1} \ar[rd]^{p_2}  &       \\
      X       &                               &Y
    }
    & with the property that for all objects and maps \\
     \xymatrix{
              &A \ar[ld]_{f_1} \ar[rd]^{f_2}  &       \\
      X       &                               &Y
    }
    & in $\cat{A}$, there exists a unique map $\bar{f}\from A \to P$ such that \\
    \xymatrix{
        &A \ar[ldd]_{f_1} \ar@{.>}[d]|{\bar{f}\vphantom{\bar{\bar{f}}}}
        \ar[rdd]^{f_2}&       \\
                &P \ar[ld]^{p_1} \ar[rd]_{p_2}                  &       \\
        X       &                                               &Y
    }
    & commutes.  The maps $p_1$ and $p_2$ are called the \demph{projections}.%
  \end{tabular}

\end{frame}

\begin{frame}
  \frametitle{Coproduct}

  \begin{tabular}{l p{7cm}}
   \xymatrix{
      X & Y
    }

    & Let $\cat{A}$ be a category and $X, Y \in \cat{A}$.
    A \demph{coproduct} of $X$ and $Y$ consists of an object $S$ and maps \\

    \xymatrix{
      X \ar[dr]_{p_1} & & Y \ar[dl]^{p_2} \\
      & S &
    }
    & with the property that for all objects and maps \\
    \xymatrix{
      X \ar[dr]_{f_1} & & Y \ar[dl]^{f_2} \\
      & A &
    }
    & in $\cat{A}$, there exists a unique map $\bar{f}\from P \to A$ such that \\
    \xymatrix{
      X \ar[rdd]_{f_1}\ar[rd]^{p_1} & & Y \ar[ldd]^{f_2} \ar[ld]_{p_2} \\
      &  P \ar@{.>}[d]|{\bar{f}} \\
      & A &
    }
    & commutes.  % The maps $p_1$ and $p_2$ are called the \demph{projections}.%
  \end{tabular}
\end{frame}

\begin{frame}
  \frametitle{Equalizer}

  \begin{tabular}{l p{7cm}}

    \xymatrix{
      X \ar@<.5ex>[r]^s \ar@<-.5ex>[r]_t & Y
    }
    & Let $\cat{A}$ be a category and $X, Y \in \cat{A}, s, t \from X \to Y$.
    A \demph{equalizer} of this diagram is a \demph{fork} \\

    \xymatrix{
      E \ar[r]^i & X \ar@<.5ex>[r]^s \ar@<-.5ex>[r]_t & Y
    }
    & with the property that for all other forks \\

    \xymatrix{
      A \ar[r]^f & X \ar@<.5ex>[r]^s \ar@<-.5ex>[r]_t & Y
    }
    & in $\cat{A}$, there exists a unique map $\bar{f}\from A \to E$ such that \\
    \xymatrix{
      A \ar[rd]^f \ar@{.>}[dd]|{\bar{f}} & & \\
      & X \ar@<.5ex>[r]^s \ar@<-.5ex>[r]_t & Y \\
      E \ar[ru]_i & &
    }
    & commutes.
  \end{tabular}
\end{frame}

\begin{frame}
  \frametitle{Coequalizer}

  \begin{tabular}{l p{7cm}}

    \xymatrix{
      X \ar@<.5ex>[r]^s \ar@<-.5ex>[r]_t & Y
    }
    & Let $\cat{A}$ be a category and $X, Y \in \cat{A}, s, t \from X \to Y$.
    A \demph{coequalizer} of this diagram is a \demph{cofork} \\

    \xymatrix{
      X \ar@<.5ex>[r]^s \ar@<-.5ex>[r]_t & Y \ar[r]^i & C
    }
    & with the property that for all other coforks \\

    \xymatrix{
      X \ar@<.5ex>[r]^s \ar@<-.5ex>[r]_t & Y \ar[r]^f & A
    }
    & in $\cat{A}$, there exists a unique map $\bar{f}\from C \to A$ such that \\
    \xymatrix{
      & & A  \\
      X \ar@<.5ex>[r]^s \ar@<-.5ex>[r]_t & Y \ar[ru]^f \ar[rd]_i \\
      & & C \ar@{.>}[uu]|{\bar{f}}
    }
    & commutes.
  \end{tabular}
\end{frame}

\begin{frame}
  \frametitle{Pullback}

  \begin{tabular}{l p{7cm}}

    \xymatrix{
      X \ar[r]^s & Z & Y\ar[l]_t \\
    }
    & Let $\cat{A}$ be a category and $X, Y, Z \in \cat{A}, s \from X \to Z, t \from Y \to Z$.
    A \demph{pullback} of this diagram is a \demph{pullback square} \\

    \xymatrix{
      P \ar[r]_{p_2} \ar[d]^{p_1} & Y \ar[d]_{t} \\
      X \ar[r]^s & Z
    }
    & with the property that for all other pull back squares \\

    \xymatrix{
      A \ar[r]_{f_2} \ar[d]^{f_1} & Y \ar[d]_{t} \\
      X \ar[r]^s & Z
    }
    & in $\cat{A}$, there exists a unique map $\bar{f}\from A \to P$ such that \\

    \xymatrix{
      A \ar[drr]^{f_2} \ar[ddr]_{f_1} \ar@{.>}[dr]|{\bar{f}} & & \\
      & P \ar[r]_{p_2} \ar[d]^{p_1} & Y \ar[d]_{t} \\
      & X \ar[r]^s & Z
    }
    & commutes.
  \end{tabular}
\end{frame}

\begin{frame}
  \frametitle{Pushout}

  \begin{tabular}{l p{7cm}}

    \xymatrix{
      X & Z \ar[l]_s \ar[r]^t & Y \\
    }
    & Let $\cat{A}$ be a category and $X, Y, Z \in \cat{A}, s \from Z \to X, t \from Y \to Z$.
    A \demph{pushout} of this diagram is a \demph{pushout square} \\

    \xymatrix{
      Z \ar[r]_s \ar[d]^t & X \ar[d]_{p_1} \\
      Y \ar[r]^{p_2} & P
    }
    & with the property that for all other pushout squares \\

    \xymatrix{
      Z \ar[r]_s \ar[d]^t & X \ar[d]_{f_1} \\
      Y \ar[r]^{f_2} & P
    }
    & in $\cat{A}$, there exists a unique map $\bar{f}\from P \to A$ such that \\

    \xymatrix{
      Z \ar[r]_s \ar[d]^t & X \ar[d]_{p_1} \ar[ddr]_{f_1} & \\
      Y \ar[r]^{p_2} \ar[drr]^{f_2}& P \ar@{.>}[dr]|{\bar{f}} & \\
      & & A  \\
    }
    & commutes.
  \end{tabular}
\end{frame}

\end{document}


% vim mode line:
% vim: ts=2 sts=2 sw=2 expandtab

%%% Local Variables:
%%% mode: latex
%%% TeX-master: t
%%% End:
