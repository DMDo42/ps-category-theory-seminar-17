\def\pathToRoot{../../}\documentclass{article}

\usepackage{nag}
\makeatletter
\@ifclassloaded{beamer}{}
{\usepackage[small,compact]{titlesec}}
\makeatother
\usepackage[utf8]{inputenc}
\usepackage[T1]{fontenc}
\usepackage{lmodern}
\usepackage{color}
\usepackage{parskip}
\usepackage{needspace}
\usepackage{microtype}
\usepackage{mathtools}
\usepackage{xifthen}
\usepackage{xpatch}
\usepackage{enumitem}
\usepackage{mdwlist}
\usepackage{bussproofs}
\EnableBpAbbreviations
\usepackage{tabu}
\usepackage{amssymb}
\usepackage{amsmath}
\usepackage{amsthm}
%grober hack, der den groben hack von parskip bei den amsthm sachen korrigiert
\begingroup
    \makeatletter
       \@for\theoremstyle:=definition,remark,plain\do{%
            \expandafter\g@addto@macro\csname th@\theoremstyle\endcsname{%
                        \addtolength\thm@preskip\parskip
             }%
        }
\endgroup
\usepackage[UKenglish]{babel}
\usepackage{xparse}
\usepackage{adjustbox}
\usepackage{geometry}
\usepackage{booktabs}
\usepackage{multicol}
\usepackage{soul}
\usepackage{calc}
\usepackage{textcase}
\usepackage{stmaryrd}
\usepackage{marvosym}
\usepackage{wasysym}
\usepackage{pifont}
\newcommand{\cmark}{\ding{51}}
\newcommand{\xmark}{\ding{55}}
\usepackage{tikz}
\usetikzlibrary{trees, backgrounds, shapes, chains, decorations.text, decorations.pathreplacing, circuits.logic.IEC, patterns, matrix}
\usepackage{tikz-qtree}
\usepackage{tikzsymbols}
\usepackage{fancyvrb}
\usepackage{fancyhdr}
\usepackage{verbatim}
\usepackage[framemethod=tikz]{mdframed}
\usepackage{lastpage}
\usepackage{pgfpages}
\usepackage{csquotes}
\usepackage{longtable}
\usepackage{ragged2e}
%\usepackage{stackengine}
\usepackage{censor}
\usepackage{expl3}
\usepackage{multirow}
\usepackage{hyperref}
\usepackage{environ}


% Package for Cateogry diagrams:

\usepackage{tikz-cd}



\ifcsdef{labelenumi}{
\renewcommand{\labelenumi}{(\alph{enumi})}
\renewcommand{\labelenumii}{(\roman{enumii})}
}{}

\input{\pathToRoot headers/definitions}



\tikzset{
    normal/.style={draw, semithick},
    n/.style={style=normal, circle, inner sep=1mm, minimum size=8mm},
    l/.style={style=normal, rounded corners=1mm, inner sep=1mm, minimum size=6mm},
    e/.style={style=normal, shorten >=1mm, shorten <=1mm, ->, >=stealth},
    syntax/.style={style=normal, ellipse, minimum height=6mm, minimum width=8mm}, % nodes in syntax trees
    inner/.style={style=normal, minimum size=4mm}, % inner leaves or root in normal trees
    leaf/.style={style=normal, circle, minimum size=4mm}, % leaves in normal trees
    te/.style={style=normal}, % edges in a tree
    be/.style={style=e, dashed} % binding edge
}

\newcommand{\syntaxtree}[1]{ % DEPRECATED - use tikzsyntaxtree
    \begin{tikzpicture}[baseline=(current bounding box.north)]
        \tikzset{grow=down}
        \tikzset{every node/.style={syntax}}
        \tikzset{edge from parent/.style=
            {te,
                edge from parent path={(\tikzparentnode) -- (\tikzchildnode)}}}
        \Tree #1
    \end{tikzpicture}
}

\newenvironment{tikzsyntaxtree}[1][]{
    \begin{tikzpicture}[baseline=(current bounding box.north), #1]
    \tikzset{grow=down}
    \tikzset{every tree node/.style={syntax}}
    \tikzset{edge from parent/.style={te, edge from parent path={(\tikzparentnode) -- (\tikzchildnode)}}}
}{
    \end{tikzpicture}
}


\newcommand{\DisplayScaledProof}{\maxsizebox{\linewidth}{!}{\DisplayProof}}
\newcommand{\DisplayTopProof}{\adjustbox{valign=t}{\DisplayProof}}
\newcommand{\DisplayScaledTopProof}{\adjustbox{valign=t}{\maxsizebox{\linewidth}{!}{\DisplayProof}}}


\newcolumntype{P}[1]{>{\RaggedRight\hspace{0pt}}p{#1}}

\newenvironment{prooftable}
{
    \begin{longtable}{>{\footnotesize}p{0.33\textwidth}>{\footnotesize}p{0.33\textwidth}|>{\footnotesize}P{0.15\textwidth}}
    \normalsize Textbeweis & \normalsize Erklärungen & \normalsize Schlussregel\\\hline
    \endhead
}
{
    \end{longtable}
}


\theoremstyle{definition}
\newtheorem*{definition*}{Definition} % Definition ohne Nummer
\newtheorem*{inferenceRule*}{Schlussregel}

\usepackage{titling}
\geometry{a4paper,left=2cm,right=2cm,top=2cm,bottom=3cm}


\newcommand{\licenseccjuliachristian}{\def\islicenseccjuliachristian{}}
\newcommand{\suppresslicense}{\def\issuppresslicense{}}


\AtBeginDocument{
    \pagestyle{fancy}
    \renewcommand{\headrulewidth}{0pt}
    \renewcommand{\footrulewidth}{1pt}
    \fancyhead{}
    \fancyfoot[C]{\thepage~/~\pageref{LastPage}}
    \fancyfoot[R]{\footnotesize exercise sheet from \\ \theauthor}

}


\newcommand{\pgbreakhere}{\Needspace*{4\baselineskip}}
\newcommand{\pgbreakHere}{\Needspace*{10\baselineskip}}
\newcommand{\pgbreakHERE}{\Needspace*{15\baselineskip}}

\newcommand{\raisedrule}[2][0em]{\leavevmode\leaders\hbox{\rule[#1]{1pt}{#2}}\hfill\kern0pt}

% inspired by http://tex.stackexchange.com/questions/242294/suppress-parskip-only-after-a-specific-paragraph
\makeatletter
\newlength\noparskip@parskip % used to store a backup of the parskip value
\newboolean{noparskip@triggered} % flag to indicate that noparskip was run in the current paragraph
\setboolean{noparskip@triggered}{false}
\newboolean{noparskip@active} % flag to indicate that parskip should be restored after this paragraph
\setboolean{noparskip@active}{false}
\let\noparskip@par\par % store a backup of the \par command
\@setpar{% redefine \par with the means of ltpar.dtx to stay compatible to enumerate and itemize
    \ifhmode% since we're counting occurrences of \par, \par\par would be a problem, so check that we are actually ending a paragraph
        \ifthenelse{\boolean{noparskip@active}}{%
            \setlength\parskip\noparskip@parskip% restore parskip
            \setboolean{noparskip@active}{false}% remember not the restore parskip again
        }{}%
        \ifthenelse{\boolean{noparskip@triggered}}{%
            \ifthenelse{\boolean{noparskip@active}}{}{
                % we are triggering noparskip and not currently in a noparskip already
                \setlength\noparskip@parskip\parskip % copy the current parskip into the backup variable
            }%
            \setboolean{noparskip@triggered}{false}% paragraph is ending, so noparskip is no longer triggered
            \setlength\parskip{0pt}% no parskip when the next paragraph begins
            \setboolean{noparskip@active}{true}% parskip must be restored by the next par
        }{}%
    \fi%
    \noparskip@par% run the original par command
}
\def\noparskip@backout{%
    \ifthenelse{\boolean{noparskip@active}}{%
        % a list is beginning and parskip is currently set to zero, wich would mess up the list
        \setlength\parskip{\noparskip@parskip}% restore parskip before the list begins
        \setboolean{noparskip@active}{false}%
    }{}%
    \setboolean{noparskip@triggered}{false}% there's no sense in keeping noparskip triggered throughout a list
}
\xpretocmd\begin{%
    \ifstrequal{#1}{enumerate}{\noparskip@backout}{}%
    \ifstrequal{#1}{itemize}{\noparskip@backout}{}%
    \ifstrequal{#1}{list}{\noparskip@backout}{}%
    \ifstrequal{#1}{proof}{\noparskip@backout}{}%
}{}{}
\def\noparskip{%
    \leavevmode% ensure that we are within a paragraph
    \setboolean{noparskip@triggered}{true}% trigger noparskip
}
\makeatother

\newcommand{\noparskipworkaround}{} % DEPRECATED and no longer needed


\newcommand{\head}[1]{
    {
        \setlength{\parskip}{0pt}
        \hrule height 1pt
        \vspace{.2cm}
        Saarland University \hfill Category Theory Seminar 2017\par
        Programming Systems Lab \hfill \small\url{https://courses.ps.uni-saarland.de/ct_ss17/}\par
        \tiny\raisedrule[0mm]{1pt}
        \vspace{2ex}
        \begin{center}
            \Large
            \textbf{#1}\par
            \raisedrule[2mm]{1pt}
        \end{center}
        \vspace{3ex}
    }
}

\newenvironment{leftframedparagraph}{\begin{mdframed}[hidealllines = true, leftline = true, innerleftmargin = 2ex, innerrightmargin = 0pt,
innertopmargin = 0pt, innerbottommargin = 2pt, skipabove=2ex, skipbelow=1ex, outerlinewidth = 0ex, innerlinewidth = 0.5ex]}{\end{mdframed}}
\newenvironment{leftframed}{\begin{mdframed}[hidealllines = true, leftline = true, innerleftmargin = 2ex, innerrightmargin = 0pt,
innertopmargin = 0pt, innerbottommargin = 0pt, skipabove=2ex, skipbelow=1ex, outerlinewidth = 0ex, innerlinewidth = 0.5ex]}{\end{mdframed}}

%%% Local Variables:
%%% mode: latex
%%% TeX-master: t
%%% End:


\newcommand{\uebunghead}[3][Exercise sheet:]{\def\sheetid{#2}\head{#1 #2\ifthenelse{\isundefined{\issolution}}{}{ \ifthenelse{\isundefined{\ismarking}}{(Possible solutions)}{(Marking)}} \\ #3}}

\licenseccjuliachristian


\newcommand{\amountofpoints}[1]{\ifstrequal{#1}{1}{1~Punkt}{#1~Punkte}}


% marking implies solution
\ifthenelse{\isundefined{\ismarking}}{}{\def\issolution{}}


%%%Environments
\newcounter{ExamExerciseCounter} % will only be used in exams, but must be defined here so ExerciseCounter can be reset when ExamExericise counts
\setcounter{ExamExerciseCounter}{0}
\newcounter{ExerciseCounter}[ExamExerciseCounter]
\setcounter{ExerciseCounter}{0}

\newcommand{\ExerciseNumber}{\sheetid.\arabic{ExerciseCounter}}
\renewcommand{\theExerciseCounter}{\ExerciseNumber}

\newcommand{\ExercisePointHook}[1]{}

%Aufgaben-Umgebung
\NewDocumentEnvironment{exercise}{od<>}{
    \refstepcounter{ExerciseCounter}
    \pgbreakhere
    \vspace{1ex}\textbf{Exercise\ \ExerciseNumber}%
    \IfNoValueF{#1}{ \emph{(#1)}}%
    \IfNoValueF{#2}{\hfill(\amountofpoints{#2})}%
    \IfNoValueF{#2}{\ExercisePointHook{#2}}%
    \noparskip\par\nopagebreak
}{
    \par
    \vspace{2ex}
}

\newcommand{\exercisesOnly}[1]{\ifthenelse{\isundefined{\issolution}}{#1}{}}

%Loesungs-Umgebung
\newenvironment{answer}
{
    \ifthenelse{\isundefined{\issolution}}
    {
        \comment
    }{
        \vspace{1ex}\textsl{Sample solution \ExerciseNumber}\noparskip\par\nopagebreak
    }
}{
    \ifthenelse{\isundefined{\issolution}}
    {
    }{
        \vspace{1ex}
        \hspace*{\fill}
    }
}

\newenvironment{marking}
{%
    \ifthenelse{\isundefined{\ismarking}}%
    {%
        \comment%
    }{%
        \color{red}
    }%
}{%
    \ifthenelse{\isundefined{\ismarking}}%
    {%
    }{%
    }%
}

\newenvironment{example}{\begin{leftframedparagraph}\paragraph{Example:}}{\end{leftframedparagraph}}
\newenvironment{hint}{\paragraph{Hint:}}{}
\newenvironment{caution}{\paragraph{Caution:}}{}
\newenvironment{definition}[1]{\begin{leftframedparagraph}\paragraph{Definition (#1):}}{\end{leftframedparagraph}}


\usepackage{upgreek}

\begin{document}

\uebunghead{Talk 7}{General Limits}
\author{Leonhard Staut}

\begin{exercise}[Limit definition]
  \begin{enumerate}
  \item Verify that terminal objects are a valid instance of the general limit definition.
  \item Verify that equalizers are a valid instance of the general limit definition.
  \end{enumerate}
\end{exercise}
\begin{answer}
  Recall that general limits are universal cones over a diagram.
  Therefore, instances of general limits are defined by the shape of their diagram.
  In the following let $\mathscr{C}$ be any category.
  \begin{enumerate}
  \item
    We choose the empty category as the shape $\bf{I}$.
    Let $D : \bf{I} \to \mathscr{C}$ a diagram. A cone over $D$ is an object $U$ with
    maps to every object in the diagram. Since the diagram is empty, the cone is just
    the object $U$.
    $U$ is universal if for any other cone $V$ in $\mathscr{C}$, there is a unique map
    $h : V \to U$.
    Any other cone $V$ is also just an object in $\mathscr{C}$.
    Therefore $U$ is a terminal object.
  \item
    As the shape $\bf{I}$ we choose the following diagram.
    \[
      \begin{tikzcd}
        I \arrow[r,shift left, "m"] \arrow[r,shift right,swap,"n"] & J
      \end{tikzcd}
    \]
    Let $D : \bf{I} \to \mathscr{C}$ a diagram.
    A cone over $D$ is an object $U$ with maps to every object in the diagram, such that all triangles commute.
    \[
      \begin{tikzcd}
        U \arrow{d}{f_I} \arrow{dr}{f_J} & \\
        D(I) \arrow[r,shift left, "Dm"] \arrow[r,shift right,swap,"Dn"] & D(J)
      \end{tikzcd}
    \]
    Specifically, the commuting condition says that $Dm \circ f_I = f_J = Dn \circ f_I$.
    Since $f_J$ is already completely determined by $f_I$, we can ignore the arrow
    and state equivalently that the following diagram commutes.
    \[
      \begin{tikzcd}
        U \arrow{r}{f_I} & D(I) \arrow[r,shift left, "Dm"] \arrow[r,shift right,swap,"Dn"] & D(J)
      \end{tikzcd}
    \]
    The arrow $f_J$ is simply left implicit.\\
    $U$ is a universal cone, i.e. a limit, if for any other cone $V$ in $\mathscr{C}$, there is a unique map
    $h : V \to U$.
    \[
      \begin{tikzcd}
        U \arrow{r}{f_I} & D(I) \arrow[r,shift left, "Dm"] \arrow[r,shift right,swap,"Dn"] & D(J) \\
        V \arrow{ur}{g_I} \arrow[dashed]{u}{h}& &
      \end{tikzcd}
    \]
    This is exactly the definition of an equalizer.
  \end{enumerate}
\end{answer}

\begin{exercise}[Pullback monic]
  Let $\mathscr{C}$ be any category and $X, Y$ objects.
  Show that a map $X \overset{f}{\to} Y$ is monic iff the following square is a pullback.
  \[
    \begin{tikzcd}
      X \arrow{r}{1}  \arrow{d}{1}  & X \arrow{d}{f}    \\
      X \arrow{r}{f}                & Y                  
    \end{tikzcd}
  \]
\end{exercise}

\begin{answer}
  $"\Leftarrow":$ \\
  Let $g_1, g_2 : Z \to X$ be two maps. If $f \circ g_1 = f \circ g_2$
  we have the following commuting diagram.
  \[
    \begin{tikzcd}
      Z \arrow[bend right=30]{ddr}{g_1} \arrow[bend left=30]{drr}{g_2}&   &\\
             & X \arrow{r}{1}  \arrow{d}{1}  & X \arrow{d}{f}    \\
             & X \arrow{r}{f}                & Y
    \end{tikzcd}
  \]
  Therefore $Z$ is a cone. Since is $X$ is a pullback we have a unique map $h : Z \to X$,
  such that the following diagram commutes.
  \[
    \begin{tikzcd}
      Z \arrow[bend right=30]{ddr}{g_1} \arrow[bend left=30]{drr}{g_2} \arrow[dashed]{dr}{h}&   &\\
             & X \arrow{r}{1}  \arrow{d}{1}  & X \arrow{d}{f}    \\
             & X \arrow{r}{f}                & Y
    \end{tikzcd}
  \]
  It follows that $1 \circ h = h = g_1 = g_2$.
  Because there is only one map $Z \to X$, $f$ is trivially monic.\\
  $"\Rightarrow":$ \\
  The following square certainly commutes because $f \circ 1 = f \circ 1$.
  \[
    \begin{tikzcd}
      X \arrow{r}{1}  \arrow{d}{1}  & X \arrow{d}{f}    \\
      X \arrow{r}{f}                & Y
    \end{tikzcd}
  \]
  Therefore it is a cone and we only need to show its universality.
  We have to show that, assuming f is monic, for any other cone $Z$ there is a unique map
  $h : Z \to X$ making the following diagram commute.
  \[
    \begin{tikzcd}
      Z \arrow[bend right=30]{ddr}{g_1} \arrow[bend left=30]{drr}{g_2} \arrow[dashed]{dr}{h}&   &\\
             & X \arrow{r}{1}  \arrow{d}{1}  & X \arrow{d}{f}    \\
             & X \arrow{r}{f}                & Y
    \end{tikzcd}
  \]
  We know the outer diagram commutes because $Z$ is a cone, so $f \circ g_1 = f \circ g_2$
  Therefore we have $g_1 = g_2$ because $f$ is monic.
  Since all the maps $Z \to X$
  are the same there is exactly one map which is unique.
\end{answer}

\begin{definition}
  {Cone Category}
  Let $\mathscr{C}$ be any category, $D : \bf{I} \to \mathscr{C}$ a diagram.
  \texttt{Cone(D)} is the category of cones in $\mathscr{C}$ over $D$.
  \begin{itemize}
  \item The objects of \texttt{Cone(D)} are cones in $\mathscr{C}$ over $D$, that is, if
    $(U \overset{f_I}{\to} D(I))_{I \in \bf{I}}$ is a cone in $\mathscr{C}$, then there is a corresponding object
    in \texttt{Cone(D)}.
  \item The morphisms of \texttt{Cone(D)} are morphisms between two cones in $\mathscr{C}$, that is,
    if $(U \overset{f_I}{\to} D(I))_{I \in \bf{I}}, (V \overset{g_I}{\to}~ D(I))_{I \in \bf{I}}$
    are two cones in $\mathscr{C}$ and there is a morphism $V \overset{m}{\to} U$, then there
    is a corresponding morphism in \texttt{Cone(D)}.
  \end{itemize}
\end{definition}

\begin{exercise}[Cone category]
  Consider an object $L \in$ \texttt{Cone(D)} that corresponds to a limit in $\mathscr{C}$ over $D$.
  What kind of object is $L$?\\
  What about an object $L' \in$ \texttt{Cone(D)} that corresponds to a colimit in $\mathscr{C}$ over $D$?
\end{exercise}
\begin{answer}
  If $L$ corresponds to a limit, then by definition any other cone with vertex $V$ in $\mathscr{C}$ has a unique map
  from $V$ to the limit that corresponds to a map in \texttt{Cone(D)}.
  Because the cones in $\mathscr{C}$ are exactly the objects in \texttt{Cone(D)},
  $L$ has a unique arrow from every object \texttt{Cone(D)}.
  Therefore $L$ is a terminal object.\\
  By duality $L'$ is an initial object.
\end{answer}

\begin{definition}
  {Constant Functor}
  For any two categories $\mathscr{C}$ and $\mathscr{D}$,
  the constant functor $\Updelta _ V : \mathscr{C} \to \mathscr{D} $ for some object $V \in \mathscr{D}$
  is defined by:
  \begin{itemize}
    \item $\forall A \in \mathscr{C}. \ \Updelta _ V (A) = V$
    \item $\forall A \overset{f}{\to} B \in \mathscr{C}. \ \Updelta _ V (f) = 1_V$
  \end{itemize}
  
\end{definition}

\begin{exercise}[Cones as natural transformations]
  Let $\mathscr{C}$ and $\bf{I}$ be any categories, $D : \bf{I} \to \mathscr{C}$ a diagram.
  Verify that a natural transformation $\pi : \Updelta_V \to D(\bf{I})$ is a cone over $D$ with vertex $V$.
\end{exercise}
\begin{answer}
  Recall that a natural transformation $\pi : \Updelta_V \to D(\bf{I})$ has components for every object $I \in \bf{I}$
  and for every morphism $m : I \to J \in \bf{I}$ the following square commutes.
  \[
    \begin{tikzcd}
      \Updelta_V (I) \arrow{r}{\pi_I}  \arrow{d}{\Updelta_V (m)}  & D(I) \arrow{d}{D(m)}    \\
      \Updelta_V (J) \arrow{r}{\pi_J}                & D(J)
    \end{tikzcd}
  \]
  Now we unfold the definition of $\Updelta_V$.
  \[
    \begin{tikzcd}
      \Updelta_V (I) \arrow{r}{\pi_I}  \arrow{d}{\Updelta_V (m)}  & D(I) \arrow{d}{D(m)}    \\
      \Updelta_V (J) \arrow{r}{\pi_J}                & D(J)
    \end{tikzcd}
    \ = \
    \begin{tikzcd}
      V \arrow{r}{\pi_I}  \arrow{d}{1_V}  & D(I) \arrow{d}{D(m)}    \\
      V \arrow{r}{\pi_J}                & D(J)
    \end{tikzcd}
    \ = \
    \begin{tikzcd}
      V \arrow{r}{\pi_I} \arrow{dr}{\pi_J} & D(I) \arrow{d}{D(m)}    \\
                     & D(J)
    \end{tikzcd}
  \]
  To summarize, the natural transformation $\pi$ has maps $\pi_I$ for all $I \in \bf{I}$,
  and the above triangle commutes for all maps $m : I \to J$.
  This is exactly the definition of a cone with vertex $V$.
  In fact this yields an alternative equivalent definition of cones.
\end{answer}

\begin{exercise}[Uniqueness of limits]
  Let $\mathscr{C}$ be any category. Show that limits in $\mathscr{C}$ are unique up to isomorphism.
\end{exercise}
\begin{answer}
  Let $U, V$ be two limits. Then there are unique morphisms $f : U \to V$ and $g : V \to U$.
  Consider the morphism $g \circ f : U \to U$. As there is a unique morphism $U \to U$, it has to be
  the identity $g \circ f = 1_U$.
  Similarly $f \circ g = 1_V$, so $U$ and $V$ are isomorphic.
\end{answer}

\begin{exercise}[Limit construction]
  Recall the construction:
  \[
    \begin{tikzcd}
      E \arrow{r}{e} &
      \displaystyle \prod_{I \in \bf{I}} D(I) \arrow[r,shift left, "s"] \arrow[r,shift right,swap,"t"] &
      \displaystyle \prod_{J \overset{q}{\to} K \in \bf{I}} D(K)
    \end{tikzcd}
  \]
  where $E$ is the equalizer of $s$ and $t$.
  We have shown that $E$ is a cone for any diagram $D$. Show that $E$ is also the limit,
  i.e. a universal cone over $D$.
\end{exercise}
\begin{hint}
  \begin{itemize}
  \item You may use the fact that $E$ is a cone iff $E$ is an equalizer of $s$ and $t$.
  \item You do not need the concrete definitions of $s$ and $t$.
  \end{itemize}
\end{hint}
\begin{answer}
  To show that $E$ is a universal cone, we need to show that for any other cone $F$, there is a unique 
  morphism $h : F \to E$.
  If $F$ is a cone, then it is also an equalizer of $s$ and $t$.
  \[
    \begin{tikzcd}
      E \arrow{r}{e} &
      \displaystyle \prod_{I \in \bf{I}} D(I) \arrow[r,shift left, "s"] \arrow[r,shift right,swap,"t"] &
      \displaystyle \prod_{J \overset{q}{\to} K \in \bf{I}} D(K) \\
      F \arrow{ur}{f} & &
    \end{tikzcd}
  \]
  Since $E$ is an equalizer, and $F$ is a cone over the same diagram, we have a unique map $h : F \to E$.
  \[
    \begin{tikzcd}
      E \arrow{r}{e} &
      \displaystyle \prod_{I \in \bf{I}} D(I) \arrow[r,shift left, "s"] \arrow[r,shift right,swap,"t"] &
      \displaystyle \prod_{J \overset{q}{\to} K \in \bf{I}} D(K) \\
      F \arrow{ur}{f} \arrow[dashed]{u}{h} & &
    \end{tikzcd}
  \]


\end{answer}

\end{document}

%%% Local Variables:
%%% mode: latex
%%% TeX-master: t
%%% End:
