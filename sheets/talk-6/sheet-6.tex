\def\pathToRoot{../../}\documentclass{article}

\usepackage{nag}
\makeatletter
\@ifclassloaded{beamer}{}
{\usepackage[small,compact]{titlesec}}
\makeatother
\usepackage[utf8]{inputenc}
\usepackage[T1]{fontenc}
\usepackage{lmodern}
\usepackage{color}
\usepackage{parskip}
\usepackage{needspace}
\usepackage{microtype}
\usepackage{mathtools}
\usepackage{xifthen}
\usepackage{xpatch}
\usepackage{enumitem}
\usepackage{mdwlist}
\usepackage{bussproofs}
\EnableBpAbbreviations
\usepackage{tabu}
\usepackage{amssymb}
\usepackage{amsmath}
\usepackage{amsthm}
%grober hack, der den groben hack von parskip bei den amsthm sachen korrigiert
\begingroup
    \makeatletter
       \@for\theoremstyle:=definition,remark,plain\do{%
            \expandafter\g@addto@macro\csname th@\theoremstyle\endcsname{%
                        \addtolength\thm@preskip\parskip
             }%
        }
\endgroup
\usepackage[UKenglish]{babel}
\usepackage{xparse}
\usepackage{adjustbox}
\usepackage{geometry}
\usepackage{booktabs}
\usepackage{multicol}
\usepackage{soul}
\usepackage{calc}
\usepackage{textcase}
\usepackage{stmaryrd}
\usepackage{marvosym}
\usepackage{wasysym}
\usepackage{pifont}
\newcommand{\cmark}{\ding{51}}
\newcommand{\xmark}{\ding{55}}
\usepackage{tikz}
\usetikzlibrary{trees, backgrounds, shapes, chains, decorations.text, decorations.pathreplacing, circuits.logic.IEC, patterns, matrix}
\usepackage{tikz-qtree}
\usepackage{tikzsymbols}
\usepackage{fancyvrb}
\usepackage{fancyhdr}
\usepackage{verbatim}
\usepackage[framemethod=tikz]{mdframed}
\usepackage{lastpage}
\usepackage{pgfpages}
\usepackage{csquotes}
\usepackage{longtable}
\usepackage{ragged2e}
%\usepackage{stackengine}
\usepackage{censor}
\usepackage{expl3}
\usepackage{multirow}
\usepackage{hyperref}
\usepackage{environ}


% Package for Cateogry diagrams:

\usepackage{tikz-cd}



\ifcsdef{labelenumi}{
\renewcommand{\labelenumi}{(\alph{enumi})}
\renewcommand{\labelenumii}{(\roman{enumii})}
}{}

\input{\pathToRoot headers/definitions}



\tikzset{
    normal/.style={draw, semithick},
    n/.style={style=normal, circle, inner sep=1mm, minimum size=8mm},
    l/.style={style=normal, rounded corners=1mm, inner sep=1mm, minimum size=6mm},
    e/.style={style=normal, shorten >=1mm, shorten <=1mm, ->, >=stealth},
    syntax/.style={style=normal, ellipse, minimum height=6mm, minimum width=8mm}, % nodes in syntax trees
    inner/.style={style=normal, minimum size=4mm}, % inner leaves or root in normal trees
    leaf/.style={style=normal, circle, minimum size=4mm}, % leaves in normal trees
    te/.style={style=normal}, % edges in a tree
    be/.style={style=e, dashed} % binding edge
}

\newcommand{\syntaxtree}[1]{ % DEPRECATED - use tikzsyntaxtree
    \begin{tikzpicture}[baseline=(current bounding box.north)]
        \tikzset{grow=down}
        \tikzset{every node/.style={syntax}}
        \tikzset{edge from parent/.style=
            {te,
                edge from parent path={(\tikzparentnode) -- (\tikzchildnode)}}}
        \Tree #1
    \end{tikzpicture}
}

\newenvironment{tikzsyntaxtree}[1][]{
    \begin{tikzpicture}[baseline=(current bounding box.north), #1]
    \tikzset{grow=down}
    \tikzset{every tree node/.style={syntax}}
    \tikzset{edge from parent/.style={te, edge from parent path={(\tikzparentnode) -- (\tikzchildnode)}}}
}{
    \end{tikzpicture}
}


\newcommand{\DisplayScaledProof}{\maxsizebox{\linewidth}{!}{\DisplayProof}}
\newcommand{\DisplayTopProof}{\adjustbox{valign=t}{\DisplayProof}}
\newcommand{\DisplayScaledTopProof}{\adjustbox{valign=t}{\maxsizebox{\linewidth}{!}{\DisplayProof}}}


\newcolumntype{P}[1]{>{\RaggedRight\hspace{0pt}}p{#1}}

\newenvironment{prooftable}
{
    \begin{longtable}{>{\footnotesize}p{0.33\textwidth}>{\footnotesize}p{0.33\textwidth}|>{\footnotesize}P{0.15\textwidth}}
    \normalsize Textbeweis & \normalsize Erklärungen & \normalsize Schlussregel\\\hline
    \endhead
}
{
    \end{longtable}
}


\theoremstyle{definition}
\newtheorem*{definition*}{Definition} % Definition ohne Nummer
\newtheorem*{inferenceRule*}{Schlussregel}

\usepackage{titling}
\geometry{a4paper,left=2cm,right=2cm,top=2cm,bottom=3cm}


\newcommand{\licenseccjuliachristian}{\def\islicenseccjuliachristian{}}
\newcommand{\suppresslicense}{\def\issuppresslicense{}}


\AtBeginDocument{
    \pagestyle{fancy}
    \renewcommand{\headrulewidth}{0pt}
    \renewcommand{\footrulewidth}{1pt}
    \fancyhead{}
    \fancyfoot[C]{\thepage~/~\pageref{LastPage}}
    \fancyfoot[R]{\footnotesize exercise sheet from \\ \theauthor}

}


\newcommand{\pgbreakhere}{\Needspace*{4\baselineskip}}
\newcommand{\pgbreakHere}{\Needspace*{10\baselineskip}}
\newcommand{\pgbreakHERE}{\Needspace*{15\baselineskip}}

\newcommand{\raisedrule}[2][0em]{\leavevmode\leaders\hbox{\rule[#1]{1pt}{#2}}\hfill\kern0pt}

% inspired by http://tex.stackexchange.com/questions/242294/suppress-parskip-only-after-a-specific-paragraph
\makeatletter
\newlength\noparskip@parskip % used to store a backup of the parskip value
\newboolean{noparskip@triggered} % flag to indicate that noparskip was run in the current paragraph
\setboolean{noparskip@triggered}{false}
\newboolean{noparskip@active} % flag to indicate that parskip should be restored after this paragraph
\setboolean{noparskip@active}{false}
\let\noparskip@par\par % store a backup of the \par command
\@setpar{% redefine \par with the means of ltpar.dtx to stay compatible to enumerate and itemize
    \ifhmode% since we're counting occurrences of \par, \par\par would be a problem, so check that we are actually ending a paragraph
        \ifthenelse{\boolean{noparskip@active}}{%
            \setlength\parskip\noparskip@parskip% restore parskip
            \setboolean{noparskip@active}{false}% remember not the restore parskip again
        }{}%
        \ifthenelse{\boolean{noparskip@triggered}}{%
            \ifthenelse{\boolean{noparskip@active}}{}{
                % we are triggering noparskip and not currently in a noparskip already
                \setlength\noparskip@parskip\parskip % copy the current parskip into the backup variable
            }%
            \setboolean{noparskip@triggered}{false}% paragraph is ending, so noparskip is no longer triggered
            \setlength\parskip{0pt}% no parskip when the next paragraph begins
            \setboolean{noparskip@active}{true}% parskip must be restored by the next par
        }{}%
    \fi%
    \noparskip@par% run the original par command
}
\def\noparskip@backout{%
    \ifthenelse{\boolean{noparskip@active}}{%
        % a list is beginning and parskip is currently set to zero, wich would mess up the list
        \setlength\parskip{\noparskip@parskip}% restore parskip before the list begins
        \setboolean{noparskip@active}{false}%
    }{}%
    \setboolean{noparskip@triggered}{false}% there's no sense in keeping noparskip triggered throughout a list
}
\xpretocmd\begin{%
    \ifstrequal{#1}{enumerate}{\noparskip@backout}{}%
    \ifstrequal{#1}{itemize}{\noparskip@backout}{}%
    \ifstrequal{#1}{list}{\noparskip@backout}{}%
    \ifstrequal{#1}{proof}{\noparskip@backout}{}%
}{}{}
\def\noparskip{%
    \leavevmode% ensure that we are within a paragraph
    \setboolean{noparskip@triggered}{true}% trigger noparskip
}
\makeatother

\newcommand{\noparskipworkaround}{} % DEPRECATED and no longer needed


\newcommand{\head}[1]{
    {
        \setlength{\parskip}{0pt}
        \hrule height 1pt
        \vspace{.2cm}
        Saarland University \hfill Category Theory Seminar 2017\par
        Programming Systems Lab \hfill \small\url{https://courses.ps.uni-saarland.de/ct_ss17/}\par
        \tiny\raisedrule[0mm]{1pt}
        \vspace{2ex}
        \begin{center}
            \Large
            \textbf{#1}\par
            \raisedrule[2mm]{1pt}
        \end{center}
        \vspace{3ex}
    }
}

\newenvironment{leftframedparagraph}{\begin{mdframed}[hidealllines = true, leftline = true, innerleftmargin = 2ex, innerrightmargin = 0pt,
innertopmargin = 0pt, innerbottommargin = 2pt, skipabove=2ex, skipbelow=1ex, outerlinewidth = 0ex, innerlinewidth = 0.5ex]}{\end{mdframed}}
\newenvironment{leftframed}{\begin{mdframed}[hidealllines = true, leftline = true, innerleftmargin = 2ex, innerrightmargin = 0pt,
innertopmargin = 0pt, innerbottommargin = 0pt, skipabove=2ex, skipbelow=1ex, outerlinewidth = 0ex, innerlinewidth = 0.5ex]}{\end{mdframed}}

%%% Local Variables:
%%% mode: latex
%%% TeX-master: t
%%% End:


\newcommand{\uebunghead}[3][Exercise sheet:]{\def\sheetid{#2}\head{#1 #2\ifthenelse{\isundefined{\issolution}}{}{ \ifthenelse{\isundefined{\ismarking}}{(Possible solutions)}{(Marking)}} \\ #3}}

\licenseccjuliachristian


\newcommand{\amountofpoints}[1]{\ifstrequal{#1}{1}{1~Punkt}{#1~Punkte}}


% marking implies solution
\ifthenelse{\isundefined{\ismarking}}{}{\def\issolution{}}


%%%Environments
\newcounter{ExamExerciseCounter} % will only be used in exams, but must be defined here so ExerciseCounter can be reset when ExamExericise counts
\setcounter{ExamExerciseCounter}{0}
\newcounter{ExerciseCounter}[ExamExerciseCounter]
\setcounter{ExerciseCounter}{0}

\newcommand{\ExerciseNumber}{\sheetid.\arabic{ExerciseCounter}}
\renewcommand{\theExerciseCounter}{\ExerciseNumber}

\newcommand{\ExercisePointHook}[1]{}

%Aufgaben-Umgebung
\NewDocumentEnvironment{exercise}{od<>}{
    \refstepcounter{ExerciseCounter}
    \pgbreakhere
    \vspace{1ex}\textbf{Exercise\ \ExerciseNumber}%
    \IfNoValueF{#1}{ \emph{(#1)}}%
    \IfNoValueF{#2}{\hfill(\amountofpoints{#2})}%
    \IfNoValueF{#2}{\ExercisePointHook{#2}}%
    \noparskip\par\nopagebreak
}{
    \par
    \vspace{2ex}
}

\newcommand{\exercisesOnly}[1]{\ifthenelse{\isundefined{\issolution}}{#1}{}}

%Loesungs-Umgebung
\newenvironment{answer}
{
    \ifthenelse{\isundefined{\issolution}}
    {
        \comment
    }{
        \vspace{1ex}\textsl{Sample solution \ExerciseNumber}\noparskip\par\nopagebreak
    }
}{
    \ifthenelse{\isundefined{\issolution}}
    {
    }{
        \vspace{1ex}
        \hspace*{\fill}
    }
}

\newenvironment{marking}
{%
    \ifthenelse{\isundefined{\ismarking}}%
    {%
        \comment%
    }{%
        \color{red}
    }%
}{%
    \ifthenelse{\isundefined{\ismarking}}%
    {%
    }{%
    }%
}

\newenvironment{example}{\begin{leftframedparagraph}\paragraph{Example:}}{\end{leftframedparagraph}}
\newenvironment{hint}{\paragraph{Hint:}}{}
\newenvironment{caution}{\paragraph{Caution:}}{}
\newenvironment{definition}[1]{\begin{leftframedparagraph}\paragraph{Definition (#1):}}{\end{leftframedparagraph}}


\begin{document}

% Use Basis x or Talk x, where x is the number of the session
\uebunghead{Talk 6}{Special Constructions in Categories}
\author{Maximilian Wuttke}

\hint{Read the slides for the definitions of the six constructions}

\section{Products and Coproducts}

\begin{exercise}[Product is unique up-to isomorphism]
  Proof that for every category $\cat{A}$ all products are isomorphic. Thus we can speak of \emph{the} product.
\end{exercise}

\begin{definition}{Arbitrary\ product}
  Let $\cat{A}$ be a category, $I$ a set, and $(X_i)_{i \in I}$ a family of objects of $\cat{A}$.
  A \demph{product} of $(X_i)_{i \in I}$ consists of an object $P$ and a family of maps
  $\Bigl(P \toby{p_i} X_i\Bigr)_{i \in I}$
  with the property that for all objects $A$ and families of maps
  $\Bigl(A \toby{f_i} X_i\Bigr)_{i \in I}$
  there exists a unique map $\bar{f}\from A \to P$ such that
  $\forall i \in I: p_i \of \bar{f} = f_i$.
\end{definition}

\begin{exercise}[Greatest lower bound]
  Define the \emph{greatest lower bound} for an family of objects in a poset and proof that it is the arbitrary product of the family when the poset is seen as a category.
\end{exercise}

\begin{definition}{Arbitrary\ coproduct}
  Let $\cat{A}$ be a category, $I$ a set, and $(X_i)_{i \in I}$ a family of objects of $\cat{A}$.
  A \demph{coproduct} of $(X_i)_{i \in I}$ consists of an object $S$ and a family of maps
  $\Bigl(X_i \toby{p_i} S\Bigr)_{i \in I}$
  with the property that for all objects $A$ and families of maps
  $\Bigl(X_i \toby{f_i} A\Bigr)_{i \in I}$
  there exists a unique map $\bar{f}\from S \to A$ such that
  $\forall i \in I: \bar{f} \of p_i = f_i$.
\end{definition}

\begin{exercise}[Least upper bound]
  Define the \emph{least upper bound} for a class of objects in a poset and proof that it is the arbitrary binary coproduct of the poset seen as a category.
\end{exercise}


\begin{exercise}[Disjoint union]
  \begin{itemize}
    \item[(a)]For an index set $I$ and a family of sets $(X_i)_{i \in I}$, refine the arbitrary disjoint union $\coprod_{i \in I}{X_i}$.
    \item[(b)]Show that this arbitrary disjoint union is indeed the arbitrary coproduct in $\Set$.
  \end{itemize}
\end{exercise}

\section{Equalizer and Coequalizer}

\begin{definition}{Minus-Core of Vector Spaces}
  Let $V, W$ be Vector Spaces over a field $K$. Let $s, t \from V \to W$ be linear functions.
  We define the linear function $s - t := \lambda v\in V. \; v(s) - t(s)$.
  Remember the definition $\forall g \from V \to W: core(g) := \set{ v \in V | g(v) = 0 }$.
  Now $core (s - t) = \set{v \in V | s(v) = t (v) }$.
\end{definition}

\begin{exercise}[Equalizer of Vector Spaces]
  Let
  $\xymatrix{
      X \ar@<.5ex>[r]^s \ar@<-.5ex>[r]_t & Y
  }$
  be a diagramm in $Vect_K$ for a field $K$. Show that $core(s-t)$ is the equalizer for this diagramm.
\end{exercise}

\begin{definition}{Reflexive,\ symmetric\ and transitive\ closure}
  Let $X$ be a arbitrary set and $R \subseteq X \times X$ a binary relation over $X$.
  We define $R_0 := R \cup \set{ (y, x) | (x, y) \in R} \cup \set{ (x, x) | x \in X}$,
  the the \emph{reflexive, symmetric closure} of R.
  Now we define inductively $\forall n \in \nat$:
  \[R_{n+1} := \set{(x_1, x_3) \in R \times R \; | \; \exists x_2 \in X, (x_1, x_2) \in R_n \land (x_2, x_3) \in R_n}.\]
  Now $S := \bigcup_{n \in \nat}{R_n}$ is the reflexive, symmetric and transitive closure of $R$.
\end{definition}

\begin{exercise}[Coequalizer of sets]
  In this exercise we will first define \emph{reflexive symmetric transitive closures} and will define the coequalizer of $\Set$.
  \begin{itemize}
    \item[(a)]First we will show that the reflexive, symmetric and transitive closure of a relation $R$ is the \emph{least} equivalence relation containing $R$.
      Let $R$ be a relation over a set $X$ and $S$ be the reflexive, symmetric and transitive closure of $R$. Show following lemmas:
      \begin{itemize}
        \item[(i)]Show $\forall n\in\nat: R_n$ and $S$ are reflexive and symmetric.
        \item[(ii)]Show $\forall n\in\nat: R \subseteq R_n$ and $R \subseteq S$.
        \item[(iii)]Show that $R_n$ is \emph{monotone}, i.\,e. $\forall n,m\in\nat: n\le m \Rightarrow R_n \subseteq R_m$.
        \item[(iv)]Now show that $S$ is transitive.
          \begin{hint}
            Let $m\in\nat$. Show $\forall x_1, x_2, x_3: (x_1, x_2) \in S \Rightarrow (x_2, x_3) \in R_m \Rightarrow (x_1, x_3) \in S$ via $\nat$-induction.
          \end{hint}
        \item[(v)]Now show that $R$ is the \emph{least} equivalence relation containing $R$, i.\,e.
          $$\forall R': \text{$R'$ is a equivalence relation} \land R \subseteq R' \Rightarrow S \subseteq R'.$$
      \end{itemize}
    \item[(b)]Now we will define the coequalizer of $\Set$.
      Let $X, Y \in \Set, s, t \from X \to Y$.
      We make the following definitions:
      \begin{alignat*}{2}
        R    &:=& \set{(s(a), t(a)) | a\in Y} \\
        \sim &:=& \text{reflexive, symmetric and transitive closure of $R$} \\
        C    &:=& Y / \sim
      \end{alignat*}
      Note that $[y] := \set{ y' \in Y | (y, y') \in \sim}$ is the equivalence class of $y$ and $Y / \sim := \set{ [y] | y \in Y}$ is the set of all equivalence classes of $\sim$.
      Also note that no equivalence class can be empty and for $[y]$ we call $y$ a representant of this class.
      We define $i \from Y \to E$ with $i(y) := [y]$.
      \begin{itemize}
        \item[(i)]Show that the following diagramm commutes:
          $\xymatrix{
            X \ar@<.5ex>[r]^s \ar@<-.5ex>[r]_t & Y \ar[r]^i & C
          }$
        \item[(ii)]Let
          $\xymatrix{
            X \ar@<.5ex>[r]^s \ar@<-.5ex>[r]_t & Y \ar[r]^f & A
          }$
          be a cummutative diagramm. Show that $f$ respects equivalence, i.\,e. $\forall (y_1, y_2) \in \sim: f(y_1) = f(y_2)$.
          \begin{hint}
            ``Destruct'' $(y_1, y_2) \in \sim$ and do a $\nat$-induction.
          \end{hint}
        \item[(iii)]Now define $\bar{f} \from C \to A$, such that the following diagramm commutes and show that there can only be one:
          \[ \xymatrix{
            & & A  \\
            X \ar@<.5ex>[r]^s \ar@<-.5ex>[r]_t & Y \ar[ru]^f \ar[rd]_i \\
            & & C \ar@{.>}[uu]|{\bar{f}}
          } \]
        \item[(iv)]Show that the fork
          $\xymatrix{
            X \ar@<.5ex>[r]^s \ar@<-.5ex>[r]_t & Y \ar[r]^i & C
          }$.
          is the coequalizer of
          $\xymatrix{
            X \ar@<.5ex>[r]^s \ar@<-.5ex>[r]_t & Y
          }$.
      \end{itemize}
  \end{itemize}
\end{exercise}

\section{Pullbacks and Pushouts}

\begin{exercise}[Pullbacks\,vs.\,producs]
  Show that if $Z$ is terminal and
  \[ \xymatrix{
    P \ar[r]_{p_2} \ar[d]^{p_1} & Y \ar[d]_{t} \\
    X \ar[r]^s & Z
  } \]
  is the pullback square in $\cat{A}$ of $X, Y, Z, s, t$ then $X, Y$ is the product in $\cat{A}$.
\end{exercise}

\begin{exercise}[Pushouts\,vs.\,coproducs]
  Show that if $Z$ is initial and
  \[ \xymatrix{
    Z \ar[r]_s \ar[d]^t & X \ar[d]_{p_1} \\
    Y \ar[r]^{p_2} & P
  } \]
  is the pullback square in $\cat{A}$ of $X, Y, Z, s, t$, then $X + Y$ is the coproduct in $\cat{A}$.
\end{exercise}

\end{document}

%%% Local Variables:
%%% mode: latex
%%% TeX-master: t
%%% End:
% vim: ts=2 sts=2 sw=2 expandtab
