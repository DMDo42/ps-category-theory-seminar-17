\def\pathToRoot{../../}\documentclass{article}

\usepackage{nag}
\makeatletter
\@ifclassloaded{beamer}{}
{\usepackage[small,compact]{titlesec}}
\makeatother
\usepackage[utf8]{inputenc}
\usepackage[T1]{fontenc}
\usepackage{lmodern}
\usepackage{color}
\usepackage{parskip}
\usepackage{needspace}
\usepackage{microtype}
\usepackage{mathtools}
\usepackage{xifthen}
\usepackage{xpatch}
\usepackage{enumitem}
\usepackage{mdwlist}
\usepackage{bussproofs}
\EnableBpAbbreviations
\usepackage{tabu}
\usepackage{amssymb}
\usepackage{amsmath}
\usepackage{amsthm}
%grober hack, der den groben hack von parskip bei den amsthm sachen korrigiert
\begingroup
    \makeatletter
       \@for\theoremstyle:=definition,remark,plain\do{%
            \expandafter\g@addto@macro\csname th@\theoremstyle\endcsname{%
                        \addtolength\thm@preskip\parskip
             }%
        }
\endgroup
\usepackage[UKenglish]{babel}
\usepackage{xparse}
\usepackage{adjustbox}
\usepackage{geometry}
\usepackage{booktabs}
\usepackage{multicol}
\usepackage{soul}
\usepackage{calc}
\usepackage{textcase}
\usepackage{stmaryrd}
\usepackage{marvosym}
\usepackage{wasysym}
\usepackage{pifont}
\newcommand{\cmark}{\ding{51}}
\newcommand{\xmark}{\ding{55}}
\usepackage{tikz}
\usetikzlibrary{trees, backgrounds, shapes, chains, decorations.text, decorations.pathreplacing, circuits.logic.IEC, patterns, matrix}
\usepackage{tikz-qtree}
\usepackage{tikzsymbols}
\usepackage{fancyvrb}
\usepackage{fancyhdr}
\usepackage{verbatim}
\usepackage[framemethod=tikz]{mdframed}
\usepackage{lastpage}
\usepackage{pgfpages}
\usepackage{csquotes}
\usepackage{longtable}
\usepackage{ragged2e}
%\usepackage{stackengine}
\usepackage{censor}
\usepackage{expl3}
\usepackage{multirow}
\usepackage{hyperref}
\usepackage{environ}


% Package for Cateogry diagrams:

\usepackage{tikz-cd}



\ifcsdef{labelenumi}{
\renewcommand{\labelenumi}{(\alph{enumi})}
\renewcommand{\labelenumii}{(\roman{enumii})}
}{}

\input{\pathToRoot headers/definitions}



\tikzset{
    normal/.style={draw, semithick},
    n/.style={style=normal, circle, inner sep=1mm, minimum size=8mm},
    l/.style={style=normal, rounded corners=1mm, inner sep=1mm, minimum size=6mm},
    e/.style={style=normal, shorten >=1mm, shorten <=1mm, ->, >=stealth},
    syntax/.style={style=normal, ellipse, minimum height=6mm, minimum width=8mm}, % nodes in syntax trees
    inner/.style={style=normal, minimum size=4mm}, % inner leaves or root in normal trees
    leaf/.style={style=normal, circle, minimum size=4mm}, % leaves in normal trees
    te/.style={style=normal}, % edges in a tree
    be/.style={style=e, dashed} % binding edge
}

\newcommand{\syntaxtree}[1]{ % DEPRECATED - use tikzsyntaxtree
    \begin{tikzpicture}[baseline=(current bounding box.north)]
        \tikzset{grow=down}
        \tikzset{every node/.style={syntax}}
        \tikzset{edge from parent/.style=
            {te,
                edge from parent path={(\tikzparentnode) -- (\tikzchildnode)}}}
        \Tree #1
    \end{tikzpicture}
}

\newenvironment{tikzsyntaxtree}[1][]{
    \begin{tikzpicture}[baseline=(current bounding box.north), #1]
    \tikzset{grow=down}
    \tikzset{every tree node/.style={syntax}}
    \tikzset{edge from parent/.style={te, edge from parent path={(\tikzparentnode) -- (\tikzchildnode)}}}
}{
    \end{tikzpicture}
}


\newcommand{\DisplayScaledProof}{\maxsizebox{\linewidth}{!}{\DisplayProof}}
\newcommand{\DisplayTopProof}{\adjustbox{valign=t}{\DisplayProof}}
\newcommand{\DisplayScaledTopProof}{\adjustbox{valign=t}{\maxsizebox{\linewidth}{!}{\DisplayProof}}}


\newcolumntype{P}[1]{>{\RaggedRight\hspace{0pt}}p{#1}}

\newenvironment{prooftable}
{
    \begin{longtable}{>{\footnotesize}p{0.33\textwidth}>{\footnotesize}p{0.33\textwidth}|>{\footnotesize}P{0.15\textwidth}}
    \normalsize Textbeweis & \normalsize Erklärungen & \normalsize Schlussregel\\\hline
    \endhead
}
{
    \end{longtable}
}


\theoremstyle{definition}
\newtheorem*{definition*}{Definition} % Definition ohne Nummer
\newtheorem*{inferenceRule*}{Schlussregel}

\usepackage{titling}
\geometry{a4paper,left=2cm,right=2cm,top=2cm,bottom=3cm}


\newcommand{\licenseccjuliachristian}{\def\islicenseccjuliachristian{}}
\newcommand{\suppresslicense}{\def\issuppresslicense{}}


\AtBeginDocument{
    \pagestyle{fancy}
    \renewcommand{\headrulewidth}{0pt}
    \renewcommand{\footrulewidth}{1pt}
    \fancyhead{}
    \fancyfoot[C]{\thepage~/~\pageref{LastPage}}
    \fancyfoot[R]{\footnotesize exercise sheet from \\ \theauthor}

}


\newcommand{\pgbreakhere}{\Needspace*{4\baselineskip}}
\newcommand{\pgbreakHere}{\Needspace*{10\baselineskip}}
\newcommand{\pgbreakHERE}{\Needspace*{15\baselineskip}}

\newcommand{\raisedrule}[2][0em]{\leavevmode\leaders\hbox{\rule[#1]{1pt}{#2}}\hfill\kern0pt}

% inspired by http://tex.stackexchange.com/questions/242294/suppress-parskip-only-after-a-specific-paragraph
\makeatletter
\newlength\noparskip@parskip % used to store a backup of the parskip value
\newboolean{noparskip@triggered} % flag to indicate that noparskip was run in the current paragraph
\setboolean{noparskip@triggered}{false}
\newboolean{noparskip@active} % flag to indicate that parskip should be restored after this paragraph
\setboolean{noparskip@active}{false}
\let\noparskip@par\par % store a backup of the \par command
\@setpar{% redefine \par with the means of ltpar.dtx to stay compatible to enumerate and itemize
    \ifhmode% since we're counting occurrences of \par, \par\par would be a problem, so check that we are actually ending a paragraph
        \ifthenelse{\boolean{noparskip@active}}{%
            \setlength\parskip\noparskip@parskip% restore parskip
            \setboolean{noparskip@active}{false}% remember not the restore parskip again
        }{}%
        \ifthenelse{\boolean{noparskip@triggered}}{%
            \ifthenelse{\boolean{noparskip@active}}{}{
                % we are triggering noparskip and not currently in a noparskip already
                \setlength\noparskip@parskip\parskip % copy the current parskip into the backup variable
            }%
            \setboolean{noparskip@triggered}{false}% paragraph is ending, so noparskip is no longer triggered
            \setlength\parskip{0pt}% no parskip when the next paragraph begins
            \setboolean{noparskip@active}{true}% parskip must be restored by the next par
        }{}%
    \fi%
    \noparskip@par% run the original par command
}
\def\noparskip@backout{%
    \ifthenelse{\boolean{noparskip@active}}{%
        % a list is beginning and parskip is currently set to zero, wich would mess up the list
        \setlength\parskip{\noparskip@parskip}% restore parskip before the list begins
        \setboolean{noparskip@active}{false}%
    }{}%
    \setboolean{noparskip@triggered}{false}% there's no sense in keeping noparskip triggered throughout a list
}
\xpretocmd\begin{%
    \ifstrequal{#1}{enumerate}{\noparskip@backout}{}%
    \ifstrequal{#1}{itemize}{\noparskip@backout}{}%
    \ifstrequal{#1}{list}{\noparskip@backout}{}%
    \ifstrequal{#1}{proof}{\noparskip@backout}{}%
}{}{}
\def\noparskip{%
    \leavevmode% ensure that we are within a paragraph
    \setboolean{noparskip@triggered}{true}% trigger noparskip
}
\makeatother

\newcommand{\noparskipworkaround}{} % DEPRECATED and no longer needed


\newcommand{\head}[1]{
    {
        \setlength{\parskip}{0pt}
        \hrule height 1pt
        \vspace{.2cm}
        Saarland University \hfill Category Theory Seminar 2017\par
        Programming Systems Lab \hfill \small\url{https://courses.ps.uni-saarland.de/ct_ss17/}\par
        \tiny\raisedrule[0mm]{1pt}
        \vspace{2ex}
        \begin{center}
            \Large
            \textbf{#1}\par
            \raisedrule[2mm]{1pt}
        \end{center}
        \vspace{3ex}
    }
}

\newenvironment{leftframedparagraph}{\begin{mdframed}[hidealllines = true, leftline = true, innerleftmargin = 2ex, innerrightmargin = 0pt,
innertopmargin = 0pt, innerbottommargin = 2pt, skipabove=2ex, skipbelow=1ex, outerlinewidth = 0ex, innerlinewidth = 0.5ex]}{\end{mdframed}}
\newenvironment{leftframed}{\begin{mdframed}[hidealllines = true, leftline = true, innerleftmargin = 2ex, innerrightmargin = 0pt,
innertopmargin = 0pt, innerbottommargin = 0pt, skipabove=2ex, skipbelow=1ex, outerlinewidth = 0ex, innerlinewidth = 0.5ex]}{\end{mdframed}}

%%% Local Variables:
%%% mode: latex
%%% TeX-master: t
%%% End:


\newcommand{\uebunghead}[3][Exercise sheet:]{\def\sheetid{#2}\head{#1 #2\ifthenelse{\isundefined{\issolution}}{}{ \ifthenelse{\isundefined{\ismarking}}{(Possible solutions)}{(Marking)}} \\ #3}}

\licenseccjuliachristian


\newcommand{\amountofpoints}[1]{\ifstrequal{#1}{1}{1~Punkt}{#1~Punkte}}


% marking implies solution
\ifthenelse{\isundefined{\ismarking}}{}{\def\issolution{}}


%%%Environments
\newcounter{ExamExerciseCounter} % will only be used in exams, but must be defined here so ExerciseCounter can be reset when ExamExericise counts
\setcounter{ExamExerciseCounter}{0}
\newcounter{ExerciseCounter}[ExamExerciseCounter]
\setcounter{ExerciseCounter}{0}

\newcommand{\ExerciseNumber}{\sheetid.\arabic{ExerciseCounter}}
\renewcommand{\theExerciseCounter}{\ExerciseNumber}

\newcommand{\ExercisePointHook}[1]{}

%Aufgaben-Umgebung
\NewDocumentEnvironment{exercise}{od<>}{
    \refstepcounter{ExerciseCounter}
    \pgbreakhere
    \vspace{1ex}\textbf{Exercise\ \ExerciseNumber}%
    \IfNoValueF{#1}{ \emph{(#1)}}%
    \IfNoValueF{#2}{\hfill(\amountofpoints{#2})}%
    \IfNoValueF{#2}{\ExercisePointHook{#2}}%
    \noparskip\par\nopagebreak
}{
    \par
    \vspace{2ex}
}

\newcommand{\exercisesOnly}[1]{\ifthenelse{\isundefined{\issolution}}{#1}{}}

%Loesungs-Umgebung
\newenvironment{answer}
{
    \ifthenelse{\isundefined{\issolution}}
    {
        \comment
    }{
        \vspace{1ex}\textsl{Sample solution \ExerciseNumber}\noparskip\par\nopagebreak
    }
}{
    \ifthenelse{\isundefined{\issolution}}
    {
    }{
        \vspace{1ex}
        \hspace*{\fill}
    }
}

\newenvironment{marking}
{%
    \ifthenelse{\isundefined{\ismarking}}%
    {%
        \comment%
    }{%
        \color{red}
    }%
}{%
    \ifthenelse{\isundefined{\ismarking}}%
    {%
    }{%
    }%
}

\newenvironment{example}{\begin{leftframedparagraph}\paragraph{Example:}}{\end{leftframedparagraph}}
\newenvironment{hint}{\paragraph{Hint:}}{}
\newenvironment{caution}{\paragraph{Caution:}}{}
\newenvironment{definition}[1]{\begin{leftframedparagraph}\paragraph{Definition (#1):}}{\end{leftframedparagraph}}


\begin{document}

% Use Basis x or Talk x, where x is the number of the session
\uebunghead{Talk 6}{Special Constructions in Categories}
\author{Maximilian Wuttke}

\hint{Read the slides for the definitions of the six constructions.}

\section{Products and Coproducts}

\begin{exercise}[Product is unique up-to isomorphism]
  Prove that for every category $\cat{A}$ and every objects $X, Y \in \cat{A}$
  all products of $X$ and $Y$ are isomorphic. Thus we can speak of \emph{the} product of $X$ and $Y$.
\end{exercise}

\begin{answer}
  \begin{proof}
    Let $X, Y \in \cat{A}$ and $P$ be a product of $X$ and $Y$ together with $p_1 \from P \to X, p_2 \from P \to Y$
    and $P'$ be a product of $X$ and $Y$ together with $p'_1 \from P' \to X, p'_2 \from P' \to Y$.

    Lets reconcider the definition of the \emph{product}:
    For every object $A \in \cat{A}$ and arrows $f_1 \from A \to X, f_2 \from A \to Y$ we get that there exists a unique map
    $\bar{f} \from A \to P$, such that
    \[ \xymatrix{
        &A \ar[ldd]_{f_1} \ar@{.>}[d]|{\bar{f}\vphantom{\bar{\bar{f}}}}
        \ar[rdd]^{f_2}&       \\
                &P \ar[ld]^{p_1} \ar[rd]_{p_2}                  &       \\
        X       &                                               &Y
    } \]
    commutes.
    Now we can instanciate $A$ with $P'$, $f_1$ with $p_1$, $f_2$ with $p_2$. By analogous instanciations we get the commuting diagramm
    \[ \xymatrix{
      & P \ar@/_/[ldd]_{p_1} \ar@/^/[rdd]^{p_2} & \\
      & P \ar[ld]^{p_1} \ar@/_/@{.>}[dd]|{\bar{f}} \ar@{.>}[u]|{f} \ar[rd]_{p_2} & \\
      X & & Y \\
      & P' \ar[lu]_{p_1'} \ar@/_/@{.>}[uu]|{\tilde{f}} \ar@{.>}[d]|{f'} \ar[ru]^{p'_2} & \\
      & P' \ar@/^/[luu]^{p'_1} \ar@/_/[ruu]_{p'_2} &
    } \]
    for exactly one $f, f', \tilde{f}, \bar{f}$. With the inner triangles we get:
    \begin{alignat*}{2}
      p_1  \of \tilde{f} &= p'_1 &\qquad p_2  \of \tilde{f} &= p'_2 \\
      p'_1 \of \bar{f}   &= p_1  &       p'_2 \of \bar{f}   &= p_2
    \end{alignat*}
    By substitution we get
    \begin{alignat*}{2}
      p_1   \of (\tilde{f} \of \bar{f}) &= p_1  &\qquad p_2  \of (\tilde{f} \of \bar{f}) &= p_2 \\
      p'_1  \of (\bar{f} \of \tilde{f}) &= p'_1 &       p'_2 \of (\bar{f} \of \tilde{f}) &= p'_2
    \end{alignat*}
    With the outer triangles we get:
    \begin{alignat*}{5}
      p_1  &\;\of&&\; f  &= p_1  &\qquad p_2  &\;\of&\; f  &= p_2 \\
      p'_1 &\;\of&&\; f' &= p'_1 &\qquad p'_2 &\;\of&\; f' &= p'_2
    \end{alignat*}
    Note that $f$ and $f'$ are \emph{unique} and the first equations hold for $f = 1_P$ and $f = \tilde{f} \of \bar{f}$, thus $\tilde{f} \of \bar{f} = 1_P$.
    The second equations holds for $f' = 1_{P'}$ and for $f' = \bar{f} \of \tilde{f}$, thus $\bar{f} \of \tilde{f} = 1_{P'}$.
    Thus $P$ and $P'$ are isomorphic. \qedhere
  \end{proof}
\end{answer}

\hint{We will now assume that all the 6 constructions are unique up-to isomorphism.}

\begin{definition}{Arbitrary\ product}
  Let $\cat{A}$ be a category, $I$ a set, and $(X_i)_{i \in I}$ a family of objects of $\cat{A}$.
  A \demph{product} of $(X_i)_{i \in I}$ consists of an object $P$ and a family of maps
  $\Bigl(P \toby{p_i} X_i\Bigr)_{i \in I}$
  with the property that for all objects $A$ and families of maps
  $\Bigl(A \toby{f_i} X_i\Bigr)_{i \in I}$
  there exists a unique map $\bar{f}\from A \to P$ such that
  $\forall i \in I: p_i \of \bar{f} = f_i$.
\end{definition}

\begin{exercise}[Greatest lower bound]
  Define the \emph{greatest lower bound} for a family of objects in a poset and prove that it is the arbitrary product of the family when the poset is seen as a category.
\end{exercise}

\begin{answer}
  Let $(A, \le)$ be a poset concidered as a category and $I$ a set.
  A \demph{lower bound} of a family $(x_i)_{i\in I}$ is an element $a \in A$, s.\,t. $\forall i \in I: a \le x_i$.
  $z \in A$ is a \demph{greatest lower bound} of this family if $\forall a \in A: \text{$a$ is a lower bound} \Rightarrow a \le z$.

  Let $z$ be a greatest lower bound of the family $(x_i)_{i\in I}$. We show that $z$ is a product of $(x_i)_{i\in I}$.
  \begin{proof}
    We first have to define a family of maps $\Bigl(z \toby{p_i} x_i\Bigr)_{i \in I}$.
    This is trivial since we have $\forall i \in I: z \le x_i$.
    Now let $a \in A$ be and objects and let $\Bigl(a \toby{f_i} x_i\Bigr)_{i \in I}$ be a family of maps from $a$.
    The existing of this maps means that $a$ is a upper bound of $(x_i)_{i\in I}$.
    Now by definition of least upper bound, $a \le z$, thus we have a arrow $\bar{f} \from a \to z$.
    All arrows are unique, thus $\bar{f}$ is unique and the diagram commutes. \qedhere
  \end{proof}
\end{answer}

\begin{definition}{Arbitrary\ coproduct}
  Let $\cat{A}$ be a category, $I$ a set, and $(X_i)_{i \in I}$ a family of objects of $\cat{A}$.
  A \demph{coproduct} of $(X_i)_{i \in I}$ consists of an object $S$ and a family of maps
  $\Bigl(X_i \toby{p_i} S\Bigr)_{i \in I}$
  with the property that for all objects $A$ and families of maps
  $\Bigl(X_i \toby{f_i} A\Bigr)_{i \in I}$
  there exists a unique map $\bar{f}\from S \to A$ such that
  $\forall i \in I: \bar{f} \of p_i = f_i$.
\end{definition}

\begin{exercise}[Disjoint union]
  \begin{itemize}
    \item[(a)]For an index set $I$ and a family of sets $(X_i)_{i \in I}$, define the arbitrary disjoint union $\coprod_{i \in I}{X_i}$.
    \item[(b)]Show that this arbitrary disjoint union is indeed the arbitrary coproduct in $\Set$.
  \end{itemize}
\end{exercise}

\begin{answer}
  First let define the arbitrary disjoint union:
  \begin{definition}{Arbitrary disjoint union}
    Let $I$ be an index set and let $(x_i)_{i \in I}$ be a family of sets.
    We define the arbitrary disjoint union of $(x_i)_{i\in I}$:
    $$\coprod_{i \in I} {x_i} := \bigcup_{i \in I} \set{ (x, i) \;\big|\; i \in I \land x \in x_i}$$
  \end{definition}
  Now let $I$ be an index set and let $(x_i)_{i\in I}$ be a family of sets. We show that $\coprod_{i \in I} {x_i}$ is the coproduct of $(x_i)_{i\in I}$.
  \begin{proof}
    We first define the projection maps: $p_i(a) := (a, i)$.
    Now let $A$ be a set and $\Bigl(x_i \toby{f_i} A\Bigr)_{i \in I}$ be a family of functions to $A$.
    We define $\bar{f} : \coprod_{i \in I} {x_i} \to A$ with $\bar{f}(a, i) := f_i(a)$.
    We have to show that $\bar{f}$ is the only function such that
    $\forall i \in I: \bar{f} \of p_i = f_i$.
    Let $i \in I$ and we check $a \in x_i$. $\bar{f}(p_i(a)) = \bar{f}(a,i) = f_i(a)$.
    Now let $\tilde{f} : \coprod_{i \in I} {x_i} \to A$ be another function sucht that
    $\forall i \in I: \tilde{f} \of p_i = f_i$.
    Let $(a, i) \in \coprod_{i \in I} {x_i}$. $\bar{f}(a,i)= f_i(a) = \tilde{f}(p_i(a)) = \tilde{f}(a,i)$.\qedhere
  \end{proof}
\end{answer}

\begin{exercise}[Direct sum of vector spaces]
  In the talk we have seen that the direct sum $V \oplus W$ of two vector spaces $V, W$ over a fixed field $K$ is the coproduct of $V$ and $W$.
  Show that it is also the product.
\end{exercise}

\begin{answer}
  \begin{proof}
    We first define the projection functions $p_1 \from V \oplus W \to V, p_2 \from V \oplus W \to W$:
    \begin{alignat*}{1}
      p_1(x,y) &:= x \\
      p_2(x,y) &:= y
    \end{alignat*}
    Now let $A$ be a vector space and $f_1 \from A \to V, f_2 \from A \to W$ be linear functions.
    We define $\bar{f}(a) := (f_1(a), f_(2))$. We clearly have
    \begin{alignat*}{2}
      p_1(\bar{f}(a)) &= p_1(f_1(a), f_2(a)) = f_1(a) \\
      p_2(\bar{f}(a)) &= p_2(f_1(a), f_2(a)) = f_2(a).
    \end{alignat*}
    Thus the diagramm
    \[ \xymatrix{
        & A \ar[ldd]_{f_1} \ar@{.>}[d]|{\bar{f}\vphantom{\bar{\bar{f}}}} \ar[rdd]^{f_2} & \\
        & {V \oplus W} \ar[ld]^{p_1} \ar[rd]_{p_2} & \\
        V & & W
    } \]
    commutes. Now we have to show that $\bar{f}$ is the only map such that the above diagramm commutes.
    Let $\tilde{f} \from A \to V \oplus W$ be a map such that $p_1 \of \tilde{f} = f_1 \land p_2 \of \tilde{f} = f_2$.
    \[
      \tilde{f}(a) = \bigl(p_1(\tilde{f}(a)), p_2(\tilde{f}(a))\bigr) = \bigl(p_1(\bar{f}(a)), p_2(\bar{f}(a))\bigr) = \bar{f}(a)\qedhere
    \]
  \end{proof}
\end{answer}

\section{Equalizer and Coequalizer}

\begin{definition}{Minus-core of linear maps in vector spaces}
  Let $V, W$ be vector spaces over a field $K$. Let $s, t \from V \to W$ be linear functions.
  We define the linear function $s - t := \lambda v\in V. \; s(v) - t(v)$.
  Remember that for $g \from V \to W$ we have defined $core(g) := \set{ v \in V | g(v) = 0 }$.
  Now $core (s - t) = \set{v \in V | s(v) = t (v) }$.
\end{definition}

\begin{exercise}[Equalizer of vector spaces]
  Let
  $\xymatrix{
      X \ar@<.5ex>[r]^s \ar@<-.5ex>[r]_t & Y
  }$
  be a diagramm in $Vect_K$ for a field $K$. Show that $core(s-t)$ is the equalizer for this diagramm.
\end{exercise}

\begin{answer}
  \begin{proof}
    Let $i \from core(s-t) \to X$ be a inclusion function. Let
    $\xymatrix{
      A \ar[r]^f & X \ar@<.5ex>[r]^s \ar@<-.5ex>[r]_t & Y
    }$
    be a commuting diagramm. Note that $f A \subseteq core(s-t)$.
    Thus we can define $\bar{f} \from A \to core(s-t)$ with $\bar{f}(a) := f(a)$.
    We see that $\forall a \in A: i(\bar{f}(a))=\bar{f}(a)=f(a)$, thus $i \of \bar{f} = f$.
    Let $\tilde{f} \from A \to core(s-t)$ be another map such that $i \of \tilde{f} = f$.
    Now $\tilde{f}(a) = i(\tilde{f}(a)) = i(\bar{f}(a)) = \bar{f}(a)$.
  \end{proof}
\end{answer}

\begin{exercise}
  Let $\parpair{X}{Y}{s}{t}$ be maps in some category.  Prove that $s = t$ if
  and only if the equalizer of $s$ and $t$ exists and is an isomorphism, if
  and only if the coequalizer of $s$ and $t$ exists and is an isomorphism.
\end{exercise}

\begin{exercise}
  Let $X$ be a set and $f\from X \to X$ a map.  Describe the coequalizer of
  $\parpairi{X}{X}{f}{1}$ in $\Set$ as explicitly as possible.
\end{exercise}

\section{Pullbacks and Pushouts}

\begin{exercise}[Pushouts\,vs.\,coproducts]
  Show that if $Z$ is initial and the pushout square
  \[ \xymatrix{
    Z \ar[r]_s \ar[d]^t & X \ar[d]_{p_1} \\
    Y \ar[r]^{p_2} & P
  } \]
  is the pushout of
  $ \xymatrix{
    X & Z \ar[l]_s \ar[r]^t & Y \\
  } $
  then $P$ is the coproduct of $X$ and $Y$.

  Remark: Note that the dual statement holds for pullbacks of terminal objects and products.
\end{exercise}

\begin{answer}
  \begin{proof}
    We show that $S := P$ is the coproduct of $X$ and $Y$ together with the projection functions $p_1$ and $p_2$.
    First note that because $Z$ is initial $s$ and $t$ are unique maps.
    Let
    $ \xymatrix{
      X \ar[r]^{f_1} & A & Y \ar[l]_{f_2}
    } $
    be a diagramm in $\cat{A}$.
    We have to show that there is a unique map $\bar{f}' \from P \to A$ shuch that
    \[ \xymatrix{
      X \ar[rdd]_{f_1}\ar[rd]^{p_1} & & Y \ar[ldd]^{f_2} \ar[ld]_{p_2} \\
      & P \ar@{.>}[d]|{\bar{f}'} & \\
      & A &
    } \]
    commutes. We know that the diagramm
    \[ \xymatrix{
      Z \ar[r]_s \ar[d]^t & X \ar[d]_{f_1} \\
      Y \ar[r]^{f_2} & A
    } \]
    commutes because $Z$ is initial.
    Now because this square is then a pushout square there exists a unique map $\bar{f}\from P \to A$ such that
    \[ \xymatrix{
      Z \ar[r]_s \ar[d]^t & X \ar[d]_{p_1} \ar[ddr]^{f_1} & \\
      Y \ar[r]^{p_2} \ar[drr]_{f_2}& P \ar@{.>}[dr]|{\bar{f}} & \\
      & & A  \\
    } \]
    commutes. We see that we can choose $\bar{f}' := \bar{f}$.
    Suppose we have another $\tilde{f} \from P \to A$ such that
    \[ \xymatrix{
      X \ar[rdd]_{f_1}\ar[rd]^{p_1} & & Y \ar[ldd]^{f_2} \ar[ld]_{p_2} \\
      & P \ar@{.>}[d]|{\tilde{f}} & \\
      & A &
    } \]
    commutes. Then
    \[ \xymatrix{
      Z \ar[r]_s \ar[d]^t & X \ar[d]_{p_1} \ar[ddr]^{f_1} & \\
      Y \ar[r]^{p_2} \ar[drr]_{f_2}& P \ar@{.>}[dr]|{\tilde{f}} & \\
      & & A  \\
    } \]
    would also commute (because $S$ is initial) but $\bar{f}$ is unique, hence $\bar{f} = \tilde{f}$.
  \end{proof}

\end{answer}

\begin{exercise}
  Take objects and maps $\xymatrix@1{E \ar[r]^i &X \ar@<.5ex>[r]^{f}
  \ar@<-.5ex>[r]_g &Y}$ in some category.  If this is an equalizer, is the
  square
  \[ \xymatrix{
    E \ar[r]^i \ar[d]_i &
    X \ar[d]^g \\
    X \ar[r]_f &
    Y
  } \]
  necessarily a pullback?
  What about the converse?  Give proofs or counterexamples.
\end{exercise}

\begin{exercise}
  Take a commutative diagram
  \[ \xymatrix{
    \cdot \ar[r] \ar[d] & \cdot \ar[r] \ar[d] & \cdot \ar[d] \\
    \cdot \ar[r] & \cdot \ar[r] & \cdot
  } \]
  in some category.  Suppose that the right-hand square is a pullback.  Show
  that the left-hand square is a pullback if and only if the outer rectangle is
  a pullback.
\end{exercise}

\end{document}

%%% Local Variables:
%%% mode: latex
%%% TeX-master: t
%%% End:
% vim: ts=2 sts=2 sw=2 expandtab
