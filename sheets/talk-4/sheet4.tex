\def\pathToRoot{../../}\documentclass{article}

\usepackage{nag}
\makeatletter
\@ifclassloaded{beamer}{}
{\usepackage[small,compact]{titlesec}}
\makeatother
\usepackage[utf8]{inputenc}
\usepackage[T1]{fontenc}
\usepackage{lmodern}
\usepackage{color}
\usepackage{parskip}
\usepackage{needspace}
\usepackage{microtype}
\usepackage{mathtools}
\usepackage{xifthen}
\usepackage{xpatch}
\usepackage{enumitem}
\usepackage{mdwlist}
\usepackage{bussproofs}
\EnableBpAbbreviations
\usepackage{tabu}
\usepackage{amssymb}
\usepackage{amsmath}
\usepackage{amsthm}
%grober hack, der den groben hack von parskip bei den amsthm sachen korrigiert
\begingroup
    \makeatletter
       \@for\theoremstyle:=definition,remark,plain\do{%
            \expandafter\g@addto@macro\csname th@\theoremstyle\endcsname{%
                        \addtolength\thm@preskip\parskip
             }%
        }
\endgroup
\usepackage[UKenglish]{babel}
\usepackage{xparse}
\usepackage{adjustbox}
\usepackage{geometry}
\usepackage{booktabs}
\usepackage{multicol}
\usepackage{soul}
\usepackage{calc}
\usepackage{textcase}
\usepackage{stmaryrd}
\usepackage{marvosym}
\usepackage{wasysym}
\usepackage{pifont}
\newcommand{\cmark}{\ding{51}}
\newcommand{\xmark}{\ding{55}}
\usepackage{tikz}
\usetikzlibrary{trees, backgrounds, shapes, chains, decorations.text, decorations.pathreplacing, circuits.logic.IEC, patterns, matrix}
\usepackage{tikz-qtree}
\usepackage{tikzsymbols}
\usepackage{fancyvrb}
\usepackage{fancyhdr}
\usepackage{verbatim}
\usepackage[framemethod=tikz]{mdframed}
\usepackage{lastpage}
\usepackage{pgfpages}
\usepackage{csquotes}
\usepackage{longtable}
\usepackage{ragged2e}
%\usepackage{stackengine}
\usepackage{censor}
\usepackage{expl3}
\usepackage{multirow}
\usepackage{hyperref}
\usepackage{environ}


% Package for Cateogry diagrams:

\usepackage{tikz-cd}



\ifcsdef{labelenumi}{
\renewcommand{\labelenumi}{(\alph{enumi})}
\renewcommand{\labelenumii}{(\roman{enumii})}
}{}

\input{\pathToRoot headers/definitions}



\tikzset{
    normal/.style={draw, semithick},
    n/.style={style=normal, circle, inner sep=1mm, minimum size=8mm},
    l/.style={style=normal, rounded corners=1mm, inner sep=1mm, minimum size=6mm},
    e/.style={style=normal, shorten >=1mm, shorten <=1mm, ->, >=stealth},
    syntax/.style={style=normal, ellipse, minimum height=6mm, minimum width=8mm}, % nodes in syntax trees
    inner/.style={style=normal, minimum size=4mm}, % inner leaves or root in normal trees
    leaf/.style={style=normal, circle, minimum size=4mm}, % leaves in normal trees
    te/.style={style=normal}, % edges in a tree
    be/.style={style=e, dashed} % binding edge
}

\newcommand{\syntaxtree}[1]{ % DEPRECATED - use tikzsyntaxtree
    \begin{tikzpicture}[baseline=(current bounding box.north)]
        \tikzset{grow=down}
        \tikzset{every node/.style={syntax}}
        \tikzset{edge from parent/.style=
            {te,
                edge from parent path={(\tikzparentnode) -- (\tikzchildnode)}}}
        \Tree #1
    \end{tikzpicture}
}

\newenvironment{tikzsyntaxtree}[1][]{
    \begin{tikzpicture}[baseline=(current bounding box.north), #1]
    \tikzset{grow=down}
    \tikzset{every tree node/.style={syntax}}
    \tikzset{edge from parent/.style={te, edge from parent path={(\tikzparentnode) -- (\tikzchildnode)}}}
}{
    \end{tikzpicture}
}


\newcommand{\DisplayScaledProof}{\maxsizebox{\linewidth}{!}{\DisplayProof}}
\newcommand{\DisplayTopProof}{\adjustbox{valign=t}{\DisplayProof}}
\newcommand{\DisplayScaledTopProof}{\adjustbox{valign=t}{\maxsizebox{\linewidth}{!}{\DisplayProof}}}


\newcolumntype{P}[1]{>{\RaggedRight\hspace{0pt}}p{#1}}

\newenvironment{prooftable}
{
    \begin{longtable}{>{\footnotesize}p{0.33\textwidth}>{\footnotesize}p{0.33\textwidth}|>{\footnotesize}P{0.15\textwidth}}
    \normalsize Textbeweis & \normalsize Erklärungen & \normalsize Schlussregel\\\hline
    \endhead
}
{
    \end{longtable}
}


\theoremstyle{definition}
\newtheorem*{definition*}{Definition} % Definition ohne Nummer
\newtheorem*{inferenceRule*}{Schlussregel}

\usepackage{titling}
\geometry{a4paper,left=2cm,right=2cm,top=2cm,bottom=3cm}


\newcommand{\licenseccjuliachristian}{\def\islicenseccjuliachristian{}}
\newcommand{\suppresslicense}{\def\issuppresslicense{}}


\AtBeginDocument{
    \pagestyle{fancy}
    \renewcommand{\headrulewidth}{0pt}
    \renewcommand{\footrulewidth}{1pt}
    \fancyhead{}
    \fancyfoot[C]{\thepage~/~\pageref{LastPage}}
    \fancyfoot[R]{\footnotesize exercise sheet from \\ \theauthor}

}


\newcommand{\pgbreakhere}{\Needspace*{4\baselineskip}}
\newcommand{\pgbreakHere}{\Needspace*{10\baselineskip}}
\newcommand{\pgbreakHERE}{\Needspace*{15\baselineskip}}

\newcommand{\raisedrule}[2][0em]{\leavevmode\leaders\hbox{\rule[#1]{1pt}{#2}}\hfill\kern0pt}

% inspired by http://tex.stackexchange.com/questions/242294/suppress-parskip-only-after-a-specific-paragraph
\makeatletter
\newlength\noparskip@parskip % used to store a backup of the parskip value
\newboolean{noparskip@triggered} % flag to indicate that noparskip was run in the current paragraph
\setboolean{noparskip@triggered}{false}
\newboolean{noparskip@active} % flag to indicate that parskip should be restored after this paragraph
\setboolean{noparskip@active}{false}
\let\noparskip@par\par % store a backup of the \par command
\@setpar{% redefine \par with the means of ltpar.dtx to stay compatible to enumerate and itemize
    \ifhmode% since we're counting occurrences of \par, \par\par would be a problem, so check that we are actually ending a paragraph
        \ifthenelse{\boolean{noparskip@active}}{%
            \setlength\parskip\noparskip@parskip% restore parskip
            \setboolean{noparskip@active}{false}% remember not the restore parskip again
        }{}%
        \ifthenelse{\boolean{noparskip@triggered}}{%
            \ifthenelse{\boolean{noparskip@active}}{}{
                % we are triggering noparskip and not currently in a noparskip already
                \setlength\noparskip@parskip\parskip % copy the current parskip into the backup variable
            }%
            \setboolean{noparskip@triggered}{false}% paragraph is ending, so noparskip is no longer triggered
            \setlength\parskip{0pt}% no parskip when the next paragraph begins
            \setboolean{noparskip@active}{true}% parskip must be restored by the next par
        }{}%
    \fi%
    \noparskip@par% run the original par command
}
\def\noparskip@backout{%
    \ifthenelse{\boolean{noparskip@active}}{%
        % a list is beginning and parskip is currently set to zero, wich would mess up the list
        \setlength\parskip{\noparskip@parskip}% restore parskip before the list begins
        \setboolean{noparskip@active}{false}%
    }{}%
    \setboolean{noparskip@triggered}{false}% there's no sense in keeping noparskip triggered throughout a list
}
\xpretocmd\begin{%
    \ifstrequal{#1}{enumerate}{\noparskip@backout}{}%
    \ifstrequal{#1}{itemize}{\noparskip@backout}{}%
    \ifstrequal{#1}{list}{\noparskip@backout}{}%
    \ifstrequal{#1}{proof}{\noparskip@backout}{}%
}{}{}
\def\noparskip{%
    \leavevmode% ensure that we are within a paragraph
    \setboolean{noparskip@triggered}{true}% trigger noparskip
}
\makeatother

\newcommand{\noparskipworkaround}{} % DEPRECATED and no longer needed


\newcommand{\head}[1]{
    {
        \setlength{\parskip}{0pt}
        \hrule height 1pt
        \vspace{.2cm}
        Saarland University \hfill Category Theory Seminar 2017\par
        Programming Systems Lab \hfill \small\url{https://courses.ps.uni-saarland.de/ct_ss17/}\par
        \tiny\raisedrule[0mm]{1pt}
        \vspace{2ex}
        \begin{center}
            \Large
            \textbf{#1}\par
            \raisedrule[2mm]{1pt}
        \end{center}
        \vspace{3ex}
    }
}

\newenvironment{leftframedparagraph}{\begin{mdframed}[hidealllines = true, leftline = true, innerleftmargin = 2ex, innerrightmargin = 0pt,
innertopmargin = 0pt, innerbottommargin = 2pt, skipabove=2ex, skipbelow=1ex, outerlinewidth = 0ex, innerlinewidth = 0.5ex]}{\end{mdframed}}
\newenvironment{leftframed}{\begin{mdframed}[hidealllines = true, leftline = true, innerleftmargin = 2ex, innerrightmargin = 0pt,
innertopmargin = 0pt, innerbottommargin = 0pt, skipabove=2ex, skipbelow=1ex, outerlinewidth = 0ex, innerlinewidth = 0.5ex]}{\end{mdframed}}

%%% Local Variables:
%%% mode: latex
%%% TeX-master: t
%%% End:


\newcommand{\uebunghead}[3][Exercise sheet:]{\def\sheetid{#2}\head{#1 #2\ifthenelse{\isundefined{\issolution}}{}{ \ifthenelse{\isundefined{\ismarking}}{(Possible solutions)}{(Marking)}} \\ #3}}

\licenseccjuliachristian


\newcommand{\amountofpoints}[1]{\ifstrequal{#1}{1}{1~Punkt}{#1~Punkte}}


% marking implies solution
\ifthenelse{\isundefined{\ismarking}}{}{\def\issolution{}}


%%%Environments
\newcounter{ExamExerciseCounter} % will only be used in exams, but must be defined here so ExerciseCounter can be reset when ExamExericise counts
\setcounter{ExamExerciseCounter}{0}
\newcounter{ExerciseCounter}[ExamExerciseCounter]
\setcounter{ExerciseCounter}{0}

\newcommand{\ExerciseNumber}{\sheetid.\arabic{ExerciseCounter}}
\renewcommand{\theExerciseCounter}{\ExerciseNumber}

\newcommand{\ExercisePointHook}[1]{}

%Aufgaben-Umgebung
\NewDocumentEnvironment{exercise}{od<>}{
    \refstepcounter{ExerciseCounter}
    \pgbreakhere
    \vspace{1ex}\textbf{Exercise\ \ExerciseNumber}%
    \IfNoValueF{#1}{ \emph{(#1)}}%
    \IfNoValueF{#2}{\hfill(\amountofpoints{#2})}%
    \IfNoValueF{#2}{\ExercisePointHook{#2}}%
    \noparskip\par\nopagebreak
}{
    \par
    \vspace{2ex}
}

\newcommand{\exercisesOnly}[1]{\ifthenelse{\isundefined{\issolution}}{#1}{}}

%Loesungs-Umgebung
\newenvironment{answer}
{
    \ifthenelse{\isundefined{\issolution}}
    {
        \comment
    }{
        \vspace{1ex}\textsl{Sample solution \ExerciseNumber}\noparskip\par\nopagebreak
    }
}{
    \ifthenelse{\isundefined{\issolution}}
    {
    }{
        \vspace{1ex}
        \hspace*{\fill}
    }
}

\newenvironment{marking}
{%
    \ifthenelse{\isundefined{\ismarking}}%
    {%
        \comment%
    }{%
        \color{red}
    }%
}{%
    \ifthenelse{\isundefined{\ismarking}}%
    {%
    }{%
    }%
}

\newenvironment{example}{\begin{leftframedparagraph}\paragraph{Example:}}{\end{leftframedparagraph}}
\newenvironment{hint}{\paragraph{Hint:}}{}
\newenvironment{caution}{\paragraph{Caution:}}{}
\newenvironment{definition}[1]{\begin{leftframedparagraph}\paragraph{Definition (#1):}}{\end{leftframedparagraph}}


\begin{document}
	\uebunghead{Talk 4}{Galois Connections and Adjunctions}
	\author {Sarah Mameche}
	
	
	\section{Galois connections}
	
	\begin{exercise}
		Let $\mathscr{P} = (P, \preceq_P)$ and $\mathscr{Q} = (Q, \preceq_Q)$ be partially ordered sets, and $F \from P \to Q, G \from Q \to P$ monotone functions with $F \dashv G$.
		\begin{enumerate}
			\item Show that the following characterizations of the Galois connection are equivalent.
			\begin{enumerate}
				\item $p \preceq_P G (F (p)) \land F(G (q)) \preceq_Q q$
				\item $p \preceq_P G(q) \Leftrightarrow F(p) \preceq_Q q$
			\end{enumerate}
			\item Show that the following property holds for $F \dashv G$. 
			\[  
			\begin{aligned}
			\forall q \in Q.\; G(q) = max\{p \in P \such F(p) \preceq_Q q\}
			\end{aligned}
			\]
			Formulate a similar statement $ \forall p \in P, F(p)$.
		\end{enumerate}
	\end{exercise}
	
	\begin{answer}
		\begin{enumerate}
			\item (i) $\Rightarrow$ (ii):
			\\Assume (i) and $p \preceq_P G(q)$. Since $F$ is monotone, $F(p) \preceq_Q F(G(q))$, and with (i):  $F(p) \preceq_Q F(G(q)) \preceq_Q q$.
			\\Assume (i) and $F(p) \preceq_Q q$. Applying monotone $G$ and using (i), $ p \preceq_P G(F(p)) \preceq_P G(q)$.\\ \\
			(ii) $\Rightarrow$ (i): \\Let $q := F(p)$. Since $F(p) \preceq_Q F(p)$, (ii) implies $p \preceq_P G (F (p))$. 
			\\Similarly, with $p := G(q)$,  (ii) implies $F(G (q)) \preceq_Q q$.
			
			\item  For some $q \in Q$, set $D = \{p \in P \such F(p) \preceq_Q q\}$. 
			\\We know from (a) that $\forall q, F(G(q)) \preceq_Q q$, so $G(q) \in D$.
			\\To show that $G(q)$ is also the maximum of D, take some $d \in D$, i.e. $F(d) \preceq_Q q$. 
			\\With monotonicity of G, $G(F(d)) \preceq_Q G(q)$. From (a), $d \preceq_Q G(F(d))$, so $d \preceq_Q G(q)$.
			\\
			\\Dual property: $\forall p \in P.\; F(p) = min\{q \in Q \such p \preceq_P G(q)\} $
		\end{enumerate}
	\end{answer}
	
	
	\begin{exercise}
		Consider the posets $\mathscr{N} = (\mathbb{N}, \leqslant), \mathscr{Q} = (\mathbb{Q}^+, \leqslant)$ 
		and the inclusion function $I : \mathscr{N} \to  \mathscr{Q}$.
		\begin{enumerate}
			\item Find Galois connections $H \dashv I$ and $I \dashv G$.
			\item Show (with the example from this exercise or a new one) that in general, Galois connections are not symmetric.
		\end{enumerate}
	\end{exercise}
	
	\begin{answer}
		\begin{enumerate} 
			\item The right adjunct $G \from \mathbb{Q}^+ \to  \mathbb{N}$ is the floor function, because for $x \in \mathbb{N}, y \in \mathbb{Q}^+:$
			\\$ x = \lfloor x \rfloor = G(I(x))  \land I(G(y)) =  \lfloor y \rfloor \leqslant y$.
			\\The left adjunct $H \from \mathbb{Q}^+ \to  \mathbb{N}$ is the ceiling function.\\ For $x \in \mathbb{N}, y \in \mathbb{Q}^+: y \leqslant \lceil y \rceil = I(H(x))  \land H(I(x)) =  \lceil x \rceil = x$.\\
			
			\item $I \dashv G$ but not $G \dashv I$. G has no right adjoint:
			\\$\{q\in \mathbb{Q}^+ \such G(q) \leqslant 1\} $ has no maximum, because $G(q)= 1 \; \forall 1 \leqslant q < 2$. 
			The statement follows with Ex.1 (b).
		\end{enumerate}
	\end{answer}
	
	
	\section{Adjunctions}
	
	\begin{exercise}
		Recall the naturality axiom that has to hold for the isomorphism $\mathscr{B}(F(A), B) \; \cong \; \mathscr{A}(A, G(B)) $:
		
		\begin{enumerate}[label=(\roman*)]
			\item for any $g \from F(A) \to B$ and $q \from B \to B': \overline{q \of g} = G(q) \of \bar{g} $
			\item for any $f \from A \to G(B) $ and $ p \from A' \to A: \overline{f \of p} = \bar{f} \of F(p)$
		\end{enumerate}
		Show that (i) and (ii) can be replaced by (iii) $\forall f, q, p. \;\overline{G(q) \of f \of p} = q \of \bar{f} \of F(p)$
	\end{exercise}
	
	\begin{answer}
		(i), (ii) $\Rightarrow$ (iii)
		\\$\forall q \from B \to B', \forall f, p. \;$\\$ q \of \bar{f} \of F(p) = q \of \overline{f \of p}\;\;$ ((ii) and naturality of composition)\\
		$\Rightarrow q \of \bar{f} \of F(p)  = q \of h\;\;$ (*)\\
		$\Leftrightarrow  q \of \bar{f} \of F(p) = \overline{G(q) \of \bar{h}}\;\;$ (i)\\
		$\Leftrightarrow  q \of \bar{f} \of F(p)= \overline{G(q) \of f \of p}\;\;$(*)\\
		In (*), $F \dashv G$ is used, so the function  $\overline{f \of p} \from A' \to G(B)$ has a transpose $ h =: f \of p \from F(A') \to B$.\\\\
		(iii) $\Rightarrow$ (i)\\
		Assume (iii) and let $p = id_A. \; F(p) = id_B$ by the functor axioms. Due to $F \dashv G$, every $f\from A \to G(B)$ has a transpose $g := \bar{f}$. Then \\\\
		$\forall q, f. \; \overline{G(q) \of f \of p} = q \of \bar{f} \of F(p)$\\
		$\Rightarrow \forall q, g.  \;\overline{G(q) \of \bar{g} \of id_A} = q \of g \of id_B$ (instantiations) \\
		$\Leftrightarrow \forall q, g.  \;\overline{G(q) \of \bar{g}} = q \of g$ (identities)\\
		$\Leftrightarrow \forall q, g. \; G(q) \of \bar{g} = \overline{q \of g}$ (transposition)\\
		
		(iii)  $\Rightarrow$ (ii) analogously.
	\end{answer}
	
	
	\begin{exercise}
		Find a right adjunct to the forgetful functor $F \from $\textbf{Set }$ \to $\textbf{ Rel}.
	\end{exercise}
	
	\begin{answer}
		The right adjunct is the powerset functor $P \from $\textbf{Rel }$ \to $ \textbf{ Set}. 
		It maps a set $A$ to its powerset $\mathcal{P}(A)$. 
		A Relation $R \subseteq A \times B$ is mapped to the function $f_R \from \mathcal{P}(A) \to \mathcal{P}(B), f_R(X) = \{b \in B \such \exists a \in X. (a,b) \in R \}$. 
		\\Every function $f \from A \to \mathcal{P}(A)$ can be seen as a relation $R_f \subseteq F(A) \times B$, with $(x,y) \in R \Leftrightarrow y \in f(x)$, and vice versa. 
		So \textbf{Rel }$(F(A), B) \cong $\textbf{ Set}$(A, P(B))$.\\
		Naturality in $A$: for functions $h \from A' \to A$,  the constructed relation for $f \of h$ is $R_{f \of h} = \{(x,y) \such  y \in f(h(x))\}.$ 
		The relation $R_f \of P(h)$ is the same one:  $R_f \of P(h)= R_f \of R_h = \{(x,z) \such \exists y. \; z \in f(y) \land y \in h(x)\} = R_{f \of h}$. 
		\\Naturality in $B$: for relations $R \subseteq B \times B', f_{B \times B'}$ is the same as $f_B \of F(B').$
	\end{answer}
	
	
	\begin{exercise}
		\begin{enumerate}
			\item Find a functor that is both left and right adjoint to another functor.
			\item Find a functor that is self-adjoint.
		\end{enumerate}
	\end{exercise}
	
	\begin{answer}
		\begin{enumerate}
			\item We have seen that initial (terminal) objects of categories correspond to left (right) adjoints. So for any category that has a zero object, there is a functor that is both left and right adjoint. Also, an equivalence between two categories is both a left and right adjoint.
			
			\item The contravariant powerset functor $\mathcal{P} \from $\textbf {Set }$ \to$\textbf{ Set}$^{op}$ (see Ex. 7) is self-adjoint. 
			As seen in Exercise 4, Functions $A \to \mathcal{P}(B)$ in \textbf{Set} correspond to subsets of $A \times B$ and thus also to functions $B \to \mathcal{P}(A)$.
		\end{enumerate}
	\end{answer}
	
	
	\begin{exercise}
		Show that adjunctions preserve initial objects. I.e., if  
		\begin{tikzcd}[ampersand replacement=\&, every label/.append style = {font = \footnotesize}]
			\mathscr{A} \arrow[r, shift left,"F"] 
			\& \mathscr{B} \arrow[l, shift left, "G"] 
		\end{tikzcd}
		with $F\dashv G$ and $I$ is initial in $ \mathscr{A}$, then $F(I)$ is initial in $\mathscr{B}$. Dualize the statement.
	\end{exercise}
	
	\begin{answer}
		Assume $I$ is initial in $ \mathscr{A}$, then $\forall A \in ob(\mathscr{A})$, the maps $I \to A$ are unique. 
		This implies that $\forall B \in ob(\mathscr{B})$, the maps  $I \to G(B)$ are unique because $G(B) \in ob(\mathscr{A})$. 
		From $F\dashv G$, we get that the corresponding maps $F(I) \to B$ are unique, so $F(I)$ initial in $\mathscr{B}$.\\\\
		Dually, terminal objects are also preserved by $G$.
	\end{answer}
	
	
	\begin{exercise}
		For sets $X, Y$ and a function $f: X \to Y$, show that the following operations are adjunctions:\\
		$ \forall A \in \mathcal{P}(X).\; F \from \mathcal {P}(X) \to \mathcal {P}(Y ) $ with $ F(A) := \{f(a) \such a \in A \}$ and\\
		$\forall B\in \mathcal{P}(Y). \; F^{-1} \from \mathcal {P}(Y) \to \mathcal {P}(X)$ with $ F^{-1}(B):=\{a\in X \such f(a)\in B\}$
		\\Adapt the construction so that it works for groups (alternatively topological spaces) instead of sets. 
	\end{exercise}
	
	\begin{answer}
		$F \dashv F^{-1}$, meaning $F(X) \subseteq Y \Leftrightarrow X \subseteq F^{-1}(Y)$.\\
		Assume $F(X) \subseteq Y$.  $\forall x \in X, f(x) \in Y$ and thus $X \subseteq F^{-1}(Y)$\\
		Assume $X \subseteq F^{-1}(Y)$, then $\forall x \; x \in F(Y)$, which means $F(X) \subseteq Y$\\
		For Groups, f is a homomorphism, and $\mathcal{P}(G)$ the set of subgroups.\\
		For Topological Spaces, we have continuous maps and subspaces.
	\end{answer}
	
	\begin{exercise}
		Consider a functor $Ob \from $\textbf {CAT }$  \to $\textbf { Set} which maps a category to its set of objects, and a functor to the mapping induced on objects. Are there left and right adjoints?
	\end{exercise}
	
	\begin{answer}
		There is a left adjoint $D$, where $D(A)$ is a discrete category with objects $A$ (for some set $A$). \\
		For a category $\mathscr{B}$, the functors in \textbf{CAT} $(D(A), \mathscr{B})$ are fully defined by the object mapping, since $D(A)$ has only identity arrows. So the functors can be seen as functions in \textbf{Set} $(A$, ob($\mathscr{B}))$.
		\\There is also a right adjoint $I$, where $I(A)$ is the category with objects $A$ and a unique arrow $a \to b$ for each $a,b \in A$. Here, functors in \textbf{CAT} $(\mathscr{B}, I(A))$ can also be seen as functions in \textbf{Set} (ob($\mathscr{B}), A)$. That is because a functor $F$ must map arrows $B \to B'$ to $F(B) \to F(B')$, and if there is a unique arrow $F(B) \to F(B')$, we need only the object mapping to describe the functor.
	\end{answer}
	
	\begin{exercise}
		Let $G$ be a group. \begin{enumerate}
			\item Find functors between \textbf{Set} and the functor category $[G,$ \textbf{Set}$]$ of left $G$-sets. Which adjunnctions are there between them?
			\item Similarly, find functors \textbf{Vect}$_k$ and $[G,$ \textbf{Vect}$_k]$ and look for adjunctions.
		\end{enumerate}
	\end{exercise}
	
	\begin{answer}
		\begin{enumerate}
			\item Objects of $[G,$ \textbf{Set}$]$ are functors $F_S$, with one object (set S) and with maps $f_g \from S\to S$ (for each $g \in G$, left group action). Maps in  $[G,$ \textbf{Set}$]$ are natural transformations between the functors. The canonical functor $F \from$ \textbf{Set}$ \to [G,$ \textbf{Set}$]$ maps a set $S$ to the functor with $S$ as the underlying set. Functions $S\to S' $ are mapped to the natural transformation between the resp. functors $F_S, F_S'$. A functor G which goes back to \textbf{Set} is defined similarly. Then $F \dashv G$ and $G \dashv F$, because every function $S \to S'$ corresponds by construction to exactly one natural transformation $\alpha \from F_S \to F_S'$. The correspondence is natural in S, S' because $\alpha$ preserves the group action: $\alpha (g . s) = g . (\alpha . s)\; \forall s \in S$ (because $\alpha$ is a natural transformation).
			
			\item The Functors $F_V$ in  $[G,$ \textbf{Vect}$_k]$ select a vector space $V$, and have linear maps $f_g: V \to W$ for each $g \in G$. Construct $F$ and $G$ as in (a), then  $F \dashv G$ and $G \dashv F$, because the natural transformations preserve the group action.
		\end{enumerate}
	\end{answer}
	
	
	
\end{document}
%%% Local Variables:
%%% mode: latex
%%% TeX-master: t
%%% End:
