\def\pathToRoot{../../}\documentclass{article}

\usepackage{nag}
\makeatletter
\@ifclassloaded{beamer}{}
{\usepackage[small,compact]{titlesec}}
\makeatother
\usepackage[utf8]{inputenc}
\usepackage[T1]{fontenc}
\usepackage{lmodern}
\usepackage{color}
\usepackage{parskip}
\usepackage{needspace}
\usepackage{microtype}
\usepackage{mathtools}
\usepackage{xifthen}
\usepackage{xpatch}
\usepackage{enumitem}
\usepackage{mdwlist}
\usepackage{bussproofs}
\EnableBpAbbreviations
\usepackage{tabu}
\usepackage{amssymb}
\usepackage{amsmath}
\usepackage{amsthm}
%grober hack, der den groben hack von parskip bei den amsthm sachen korrigiert
\begingroup
    \makeatletter
       \@for\theoremstyle:=definition,remark,plain\do{%
            \expandafter\g@addto@macro\csname th@\theoremstyle\endcsname{%
                        \addtolength\thm@preskip\parskip
             }%
        }
\endgroup
\usepackage[UKenglish]{babel}
\usepackage{xparse}
\usepackage{adjustbox}
\usepackage{geometry}
\usepackage{booktabs}
\usepackage{multicol}
\usepackage{soul}
\usepackage{calc}
\usepackage{textcase}
\usepackage{stmaryrd}
\usepackage{marvosym}
\usepackage{wasysym}
\usepackage{pifont}
\newcommand{\cmark}{\ding{51}}
\newcommand{\xmark}{\ding{55}}
\usepackage{tikz}
\usetikzlibrary{trees, backgrounds, shapes, chains, decorations.text, decorations.pathreplacing, circuits.logic.IEC, patterns, matrix}
\usepackage{tikz-qtree}
\usepackage{tikzsymbols}
\usepackage{fancyvrb}
\usepackage{fancyhdr}
\usepackage{verbatim}
\usepackage[framemethod=tikz]{mdframed}
\usepackage{lastpage}
\usepackage{pgfpages}
\usepackage{csquotes}
\usepackage{longtable}
\usepackage{ragged2e}
%\usepackage{stackengine}
\usepackage{censor}
\usepackage{expl3}
\usepackage{multirow}
\usepackage{hyperref}
\usepackage{environ}


% Package for Cateogry diagrams:

\usepackage{tikz-cd}



\ifcsdef{labelenumi}{
\renewcommand{\labelenumi}{(\alph{enumi})}
\renewcommand{\labelenumii}{(\roman{enumii})}
}{}

\input{\pathToRoot headers/definitions}



\tikzset{
    normal/.style={draw, semithick},
    n/.style={style=normal, circle, inner sep=1mm, minimum size=8mm},
    l/.style={style=normal, rounded corners=1mm, inner sep=1mm, minimum size=6mm},
    e/.style={style=normal, shorten >=1mm, shorten <=1mm, ->, >=stealth},
    syntax/.style={style=normal, ellipse, minimum height=6mm, minimum width=8mm}, % nodes in syntax trees
    inner/.style={style=normal, minimum size=4mm}, % inner leaves or root in normal trees
    leaf/.style={style=normal, circle, minimum size=4mm}, % leaves in normal trees
    te/.style={style=normal}, % edges in a tree
    be/.style={style=e, dashed} % binding edge
}

\newcommand{\syntaxtree}[1]{ % DEPRECATED - use tikzsyntaxtree
    \begin{tikzpicture}[baseline=(current bounding box.north)]
        \tikzset{grow=down}
        \tikzset{every node/.style={syntax}}
        \tikzset{edge from parent/.style=
            {te,
                edge from parent path={(\tikzparentnode) -- (\tikzchildnode)}}}
        \Tree #1
    \end{tikzpicture}
}

\newenvironment{tikzsyntaxtree}[1][]{
    \begin{tikzpicture}[baseline=(current bounding box.north), #1]
    \tikzset{grow=down}
    \tikzset{every tree node/.style={syntax}}
    \tikzset{edge from parent/.style={te, edge from parent path={(\tikzparentnode) -- (\tikzchildnode)}}}
}{
    \end{tikzpicture}
}


\newcommand{\DisplayScaledProof}{\maxsizebox{\linewidth}{!}{\DisplayProof}}
\newcommand{\DisplayTopProof}{\adjustbox{valign=t}{\DisplayProof}}
\newcommand{\DisplayScaledTopProof}{\adjustbox{valign=t}{\maxsizebox{\linewidth}{!}{\DisplayProof}}}


\newcolumntype{P}[1]{>{\RaggedRight\hspace{0pt}}p{#1}}

\newenvironment{prooftable}
{
    \begin{longtable}{>{\footnotesize}p{0.33\textwidth}>{\footnotesize}p{0.33\textwidth}|>{\footnotesize}P{0.15\textwidth}}
    \normalsize Textbeweis & \normalsize Erklärungen & \normalsize Schlussregel\\\hline
    \endhead
}
{
    \end{longtable}
}


\theoremstyle{definition}
\newtheorem*{definition*}{Definition} % Definition ohne Nummer
\newtheorem*{inferenceRule*}{Schlussregel}

\usepackage{titling}
\geometry{a4paper,left=2cm,right=2cm,top=2cm,bottom=3cm}


\newcommand{\licenseccjuliachristian}{\def\islicenseccjuliachristian{}}
\newcommand{\suppresslicense}{\def\issuppresslicense{}}


\AtBeginDocument{
    \pagestyle{fancy}
    \renewcommand{\headrulewidth}{0pt}
    \renewcommand{\footrulewidth}{1pt}
    \fancyhead{}
    \fancyfoot[C]{\thepage~/~\pageref{LastPage}}
    \fancyfoot[R]{\footnotesize exercise sheet from \\ \theauthor}

}


\newcommand{\pgbreakhere}{\Needspace*{4\baselineskip}}
\newcommand{\pgbreakHere}{\Needspace*{10\baselineskip}}
\newcommand{\pgbreakHERE}{\Needspace*{15\baselineskip}}

\newcommand{\raisedrule}[2][0em]{\leavevmode\leaders\hbox{\rule[#1]{1pt}{#2}}\hfill\kern0pt}

% inspired by http://tex.stackexchange.com/questions/242294/suppress-parskip-only-after-a-specific-paragraph
\makeatletter
\newlength\noparskip@parskip % used to store a backup of the parskip value
\newboolean{noparskip@triggered} % flag to indicate that noparskip was run in the current paragraph
\setboolean{noparskip@triggered}{false}
\newboolean{noparskip@active} % flag to indicate that parskip should be restored after this paragraph
\setboolean{noparskip@active}{false}
\let\noparskip@par\par % store a backup of the \par command
\@setpar{% redefine \par with the means of ltpar.dtx to stay compatible to enumerate and itemize
    \ifhmode% since we're counting occurrences of \par, \par\par would be a problem, so check that we are actually ending a paragraph
        \ifthenelse{\boolean{noparskip@active}}{%
            \setlength\parskip\noparskip@parskip% restore parskip
            \setboolean{noparskip@active}{false}% remember not the restore parskip again
        }{}%
        \ifthenelse{\boolean{noparskip@triggered}}{%
            \ifthenelse{\boolean{noparskip@active}}{}{
                % we are triggering noparskip and not currently in a noparskip already
                \setlength\noparskip@parskip\parskip % copy the current parskip into the backup variable
            }%
            \setboolean{noparskip@triggered}{false}% paragraph is ending, so noparskip is no longer triggered
            \setlength\parskip{0pt}% no parskip when the next paragraph begins
            \setboolean{noparskip@active}{true}% parskip must be restored by the next par
        }{}%
    \fi%
    \noparskip@par% run the original par command
}
\def\noparskip@backout{%
    \ifthenelse{\boolean{noparskip@active}}{%
        % a list is beginning and parskip is currently set to zero, wich would mess up the list
        \setlength\parskip{\noparskip@parskip}% restore parskip before the list begins
        \setboolean{noparskip@active}{false}%
    }{}%
    \setboolean{noparskip@triggered}{false}% there's no sense in keeping noparskip triggered throughout a list
}
\xpretocmd\begin{%
    \ifstrequal{#1}{enumerate}{\noparskip@backout}{}%
    \ifstrequal{#1}{itemize}{\noparskip@backout}{}%
    \ifstrequal{#1}{list}{\noparskip@backout}{}%
    \ifstrequal{#1}{proof}{\noparskip@backout}{}%
}{}{}
\def\noparskip{%
    \leavevmode% ensure that we are within a paragraph
    \setboolean{noparskip@triggered}{true}% trigger noparskip
}
\makeatother

\newcommand{\noparskipworkaround}{} % DEPRECATED and no longer needed


\newcommand{\head}[1]{
    {
        \setlength{\parskip}{0pt}
        \hrule height 1pt
        \vspace{.2cm}
        Saarland University \hfill Category Theory Seminar 2017\par
        Programming Systems Lab \hfill \small\url{https://courses.ps.uni-saarland.de/ct_ss17/}\par
        \tiny\raisedrule[0mm]{1pt}
        \vspace{2ex}
        \begin{center}
            \Large
            \textbf{#1}\par
            \raisedrule[2mm]{1pt}
        \end{center}
        \vspace{3ex}
    }
}

\newenvironment{leftframedparagraph}{\begin{mdframed}[hidealllines = true, leftline = true, innerleftmargin = 2ex, innerrightmargin = 0pt,
innertopmargin = 0pt, innerbottommargin = 2pt, skipabove=2ex, skipbelow=1ex, outerlinewidth = 0ex, innerlinewidth = 0.5ex]}{\end{mdframed}}
\newenvironment{leftframed}{\begin{mdframed}[hidealllines = true, leftline = true, innerleftmargin = 2ex, innerrightmargin = 0pt,
innertopmargin = 0pt, innerbottommargin = 0pt, skipabove=2ex, skipbelow=1ex, outerlinewidth = 0ex, innerlinewidth = 0.5ex]}{\end{mdframed}}

%%% Local Variables:
%%% mode: latex
%%% TeX-master: t
%%% End:


\newcommand{\uebunghead}[3][Exercise sheet:]{\def\sheetid{#2}\head{#1 #2\ifthenelse{\isundefined{\issolution}}{}{ \ifthenelse{\isundefined{\ismarking}}{(Possible solutions)}{(Marking)}} \\ #3}}

\licenseccjuliachristian


\newcommand{\amountofpoints}[1]{\ifstrequal{#1}{1}{1~Punkt}{#1~Punkte}}


% marking implies solution
\ifthenelse{\isundefined{\ismarking}}{}{\def\issolution{}}


%%%Environments
\newcounter{ExamExerciseCounter} % will only be used in exams, but must be defined here so ExerciseCounter can be reset when ExamExericise counts
\setcounter{ExamExerciseCounter}{0}
\newcounter{ExerciseCounter}[ExamExerciseCounter]
\setcounter{ExerciseCounter}{0}

\newcommand{\ExerciseNumber}{\sheetid.\arabic{ExerciseCounter}}
\renewcommand{\theExerciseCounter}{\ExerciseNumber}

\newcommand{\ExercisePointHook}[1]{}

%Aufgaben-Umgebung
\NewDocumentEnvironment{exercise}{od<>}{
    \refstepcounter{ExerciseCounter}
    \pgbreakhere
    \vspace{1ex}\textbf{Exercise\ \ExerciseNumber}%
    \IfNoValueF{#1}{ \emph{(#1)}}%
    \IfNoValueF{#2}{\hfill(\amountofpoints{#2})}%
    \IfNoValueF{#2}{\ExercisePointHook{#2}}%
    \noparskip\par\nopagebreak
}{
    \par
    \vspace{2ex}
}

\newcommand{\exercisesOnly}[1]{\ifthenelse{\isundefined{\issolution}}{#1}{}}

%Loesungs-Umgebung
\newenvironment{answer}
{
    \ifthenelse{\isundefined{\issolution}}
    {
        \comment
    }{
        \vspace{1ex}\textsl{Sample solution \ExerciseNumber}\noparskip\par\nopagebreak
    }
}{
    \ifthenelse{\isundefined{\issolution}}
    {
    }{
        \vspace{1ex}
        \hspace*{\fill}
    }
}

\newenvironment{marking}
{%
    \ifthenelse{\isundefined{\ismarking}}%
    {%
        \comment%
    }{%
        \color{red}
    }%
}{%
    \ifthenelse{\isundefined{\ismarking}}%
    {%
    }{%
    }%
}

\newenvironment{example}{\begin{leftframedparagraph}\paragraph{Example:}}{\end{leftframedparagraph}}
\newenvironment{hint}{\paragraph{Hint:}}{}
\newenvironment{caution}{\paragraph{Caution:}}{}
\newenvironment{definition}[1]{\begin{leftframedparagraph}\paragraph{Definition (#1):}}{\end{leftframedparagraph}}


\begin{document}

\author{Joachim Bard, Maximilian Wuttke, Nikita Ziuzin}

% Use Basis x or Talk x, where x is the number of the session
\uebunghead{Basis 3}{Maps between Functors -- Natural Transformations}

\begin{hint}
  Read Chapter 1.3.
\end{hint}

\begin{exercise}
  Define categories $\cat{A}$ and $\cat{B}$ such that either $\cat{A} \iso \cat{B}$ in $\CAT$ or $\cat{A} \eqv \cat{B}$, but not both.
  Do not use the example given during the lecture.
\end{exercise}
\begin{answer}
    \begin{minipage}{0.5\textwidth}
        $\cat{A}$:
        \[
            \begin{tikzcd}
                A \arrow{d}{f}\\
                B
            \end{tikzcd}
        \]
    \end{minipage}
    \begin{minipage}{0.5\textwidth}
        $\cat{B}$:
        \[
            \begin{tikzcd}
                A' \arrow{dr}{f'} \arrow[bend left]{rr}{g} &    & A'' \arrow{dl}{f''} \arrow{ll}{g^{-1}} \\
                                                           & B' &
            \end{tikzcd}
        \]
    \end{minipage}
    The identity morphisms have been left out for simplicity.
    First of all note that there is no bijection between the objects of $\cat{A}$ and $\cat{B}$, so $\cat{A} \not\iso \cat{B}$.\\
    We now show $\cat{A} \eqv \cat{B}$.
    Therefore we define a functor $F$:
    \begin{align*}
        F(A) := A'\\
        F(B) := B'\\
        F(f) := f'
    \end{align*}
    In addition $F$ maps the identity morhisms to their respective identity morphisms.
    It is easy to check that $F$ is full and faithful.
    Furthermore $F$ is essentially surjective on objects because $A'$ and $A''$ are isomorphic in $\cat{B}$.
    Thus $\cat{A} \eqv \cat{B}$.
\end{answer}

\begin{exercise}
        Let $\cat{A}$ and $\cat{B}$ be categories. Prove that
        $\ftrcat{\cat{A}^\op}{\cat{B}^\op} \iso \ftrcat{\cat{A}}{\cat{B}}^\op$.
\end{exercise}

\begin{definition}{Canonical according to Leinster}
  "Canonical is an informal word, meaning something like `God-given' or
  `defined without making arbitrary choices'.  For example, for any two sets
  $A$ and $B$, there is a canonical bijection $A \times B \to B \times A$
  defined by $(a, b) \mapsto (b, a)$, and there is a canonical function $A
  \times B \to A$ defined by $(a, b) \mapsto a$.  But the function $B \to A$
  defined by `choose an element $a_0 \in A$ and send everything to $a_0$' is
  not canonical, because the choice of $a_0$ is arbitrary."
\end{definition}

\begin{exercise}
  Let $A$ and $B$ be sets, and denote by $B^A$ the set of functions from $A$
  to $B$.  Write down:
  \begin{itemize}
  \item[(a)] a canonical function $A \to B^{(B^A)}$,
  \item[(b)] a canonical function $A \times B^{A} \to B$.
  \end{itemize}
\end{exercise}
\begin{answer}
    \begin{itemize}
        \item[(a)] $\lambda\, a.\, \lambda\,f.\, f\,a$
        \item[(b)] $\lambda\, (a, f).\, f\,a$
    \end{itemize}
\end{answer}

\begin{definition}{Product of Morphisms inside a Category}
  Let $\cat{A}$ be a category and $A, B, C, D \in \ob(\cat{A})$ with $f \from A \to C$ and $g \from B \to D$.
  The product morphism of $f$ and $g$, written $f \times g$, is a function
  \begin{align*}
    A \times B &\qquad\to\qquad C \times D\\
    (a, b) &\qquad\mapsto\qquad (f(a), g(b))
  \end{align*}
\end{definition}

\newcommand{\eval}[1]{\ensuremath{\mathsf{eval}^{#1}\xspace}}

\begin{exercise}
  For a fixed set $A$, the map taking $B$ to $B^A \times A$ can be extended to
  a functor $F_A\from \mathbf{Set} \to \mathbf{Set}$ as follows:
  \begin{align*}
    F_A(B) & =  B^A \times A\\
    F_A(f: B \to C) & =  (f \circ \_) \times 1_A
  \end{align*}
  The function $\eval{A} \from F_A \to 1_{\mathbf{Set}}$, defined as $\eval{A}(f,a) := f(a)$, is a natural transformation if the following diagram commutes for each $g \from B \to C$.
\[
  \begin{tikzcd}
    F_A(B) = B^A \times A \arrow{rrr}{F_A(g)=(g\circ \_) \times 1_A} \arrow{dd}[swap]{\eval{A}_B} & & & 1_{\mathbf{Set}}(B) = B \arrow{dd}{\eval{A}_C} \\
                                                                                           & & & \\
    F_A(C) = C^A \times A \arrow{rrr}[swap]{1_{\mathbf{Set}}(g) = g} & & & 1_{\mathbf{Set}}(C) = C
  \end{tikzcd}
\]

Check that this diagram commutes.
\end{exercise}
\begin{answer}
  Let $(x,y) \in B^A \times A$. We have to show that $\eval{A}_C (F_A(g) (x,y)) = 1_{\mathbf {Set}}(g) (\eval{A}_B (x, y))$.
  By unfolding the definitions of $\times$, $\eval{A}$ and $F_A$ we get:
  \[
    \eval{A}_C (F_A(g) (x,y)) =
    \eval{A}_C ((g \circ \_)\,x, 1_A\,y) =
    \eval{A}_C ((g \circ x, y)) =
    g (x (y))
    \]
  By unfolding the definitions of $1_{\mathbf {Set}}$ and $\eval{A}$ we get:
  \[
    1_{\mathbf {Set}}(g) (\eval{A}_B (x, y)) =
    g (\eval{A}_B (x, y)) =
    g (x (y))
    \]
\end{answer}

\begin{exercise}
        Let $\xymatrix@1{\cat{A}\rtwocell^F_G{\alpha} &\cat{B}}$ be a natural transformation.
        \begin{itemize}
            \item[(a)] Suppose $\alpha$ is a natural isomorphism.
                Show that $\alpha_A\from F(A) \to G(A)$ is an isomorphism for all $A \in \cat{A}$.
            \item[(b)] Now assume that $\alpha_A\from F(A) \to G(A)$ is an isomorphism for all
                $A \in \cat{A}$. Prove that $\alpha$ is a natural isomorphism.
        \end{itemize}
\end{exercise}
\begin{answer}
    \begin{itemize}
        \item[(a)]
            By definition of natural isomorphism we have a natural transformation $\beta\from G \to F$, such that
            \[\beta \circ \alpha = 1_{F} \text{ and } \alpha \circ \beta = 1_{G}. \]
            We show that $\forall A \in \cat{A}.\, \beta_A \circ \alpha_A = 1_{F(A)} \text{ and } \alpha_A \circ \beta_A = 1_{G(A)}$.
            Suppose $A \in \cat{A}$.
            We know by definition of the identity natural transformation that $(\beta \circ \alpha)_A = 1_{F(A)}$.
            The claim $\beta_A \circ \alpha_A = 1_{F(A)}$ follows by definition of composition of natural transformations.
            $\alpha_A \circ \beta_A = 1_{G(A)}$ is shown analogously.
        \item[(b)]
            We know that for all $A \in \cat{A}$ there is a $\beta_A$ such that
            \[\beta_A \circ \alpha_A = 1_{F(A)} \text{ and } \alpha_A \circ \beta_A = 1_{G(A)}.\]
            First we prove that $\beta$ is a natural transformation.
            Suppose $f\from A \to A'$.
            We know that $\alpha$ is a natural transformation:
            \begin{alignat*}{2}
                & & G(f) \circ \alpha_A & = \alpha_{A'} \circ F(f) \\
                &\Rightarrow &\quad G(f) \circ \alpha_A \circ \beta_A & = \alpha_{A'} \circ F(f) \circ \beta_A\\
                &\Rightarrow & G(f) & = \alpha_{A'} \circ F(f) \circ \beta_A\\
                &\Rightarrow & \beta_{A'} \circ G(f) & = \beta_{A'} \circ \alpha_{A'} \circ F(f) \circ \beta_A\\
                &\Rightarrow & \beta_{A'} \circ G(f) & = F(f) \circ \beta_A
            \end{alignat*}
            By similar reasoning to (a) we get that $\beta$ is the inverse of $\alpha$.
            Therefore $\alpha$ is a natural isomorpism.
    \end{itemize}
\end{answer}

\begin{exercise}
A \emph{permutation} of a set $X$ is a bijection $X \to X$.
Write $\Sym(X)$ for the set of permutations of $X$.
A \emph{total order} on a set $X$ is an order $\leq$ such that for all $x, y \in X$,
either $x \leq y$ or $y \leq x$.
Write $\Ord(X)$ for the set of total orders on $X$.

Let $\cat{B}$ denote the category of finite sets and bijections.

        \begin{itemize}
        \item[(a)]
Give a definition of $\Sym$ on maps in $\cat{B}$ in such a way that $\Sym$
becomes a functor $\cat{B} \to \Set$.  Do the same for $\Ord$.  Both your
definitions should be canonical (no arbitrary choices).

        \item[(b)]
Show that there is no natural transformation $\Sym \to \Ord$.
(Hint: consider identity permutations.)

        \item[(c)]
For an $n$-element set $X$, how many elements do the sets $\Sym(X)$
and $\Ord(X)$ have?
        \end{itemize}

Conclude that $\Sym(X) \iso \Ord(X)$ for all $X \in \cat{B}$, but not
\emph{naturally} in $X \in \cat{B}$.  (The moral is that for each finite
set $X$, there are exactly as many permutations of $X$ as there are total
orders on $X$, but there is no natural way of matching them up.)
\end{exercise}

\begin{answer}
  \begin{itemize}
    \item[(a)]
      $\Sym(f) = \lambda g.f \circ g$, where $g$ is a permutation\\
      $\Ord(f) = \lambda (a, b).(f(a), f(b))$, where $a, b \in X$
    \item[(b)]
      Let's consider a non-identity permutation $h$. We can build an ordering
      from it by taking $(x, h(x))$ as an element of ordering relation for
      $\forall x \in \text{ some finite set } X$, then filling the gaps by
      reflexivity and transitivity.

      However, it's not the case for an identity permutation $i$, as for that
      permutation the $i(x) = x$ which leaves us only with reflexivity pairs if
      we try to build an ordering from that bijection. Thus, we cannot define
      formation of an ordering from an identity permutation and therefore
      there's no natural transformation from $\Sym$ to $\Ord$.
    \item[(c)]
      $|\Sym(X)| = n$!\\
      $|\Ord(X)| = n$!
  \end{itemize}
  We have seen that there's no natural transformation $\Sym \to \Ord$.
  Nonetheless, we have also seen that they have the same number of
  elements and an isomorphism between them can actually be formed: it's not
  hard to construct all permutations or orderings of some finite set knowing
  its elements.

  This leads us to the conclusion that $\Sym(X) \iso \Ord(X)$ for all $X \in
  \cat{B}$, but not \emph{naturally} in $X \in \cat{B}$.
\end{answer}

\begin{exercise}
Let $F\from \cat{A} \to \cat{B}$ be a functor.

        \begin{itemize}
        \item[(a)]
Suppose that $F$ is an equivalence.  Prove that $F$ is full, faithful and
essentially surjective on objects.  (Hint: prove faithfulness before
fullness.)

        \item[(b)]
Now suppose instead that $F$ is full, faithful and essentially surjective
on objects.  For each $B \in \cat{B}$, choose an object $G(B)$ of $\cat{A}$
and an isomorphism $\epsln_B\from F(G(B)) \to B$.  Prove that $G$ extends
to a functor in such a way that $(\epsln_B)_{B \in \cat{B}}$ is a natural
isomorphism $F \circ G \to 1_{\cat{B}}$.  Then construct a natural isomorphism
$1_{\cat{A}} \to G \circ F$, thus proving that $F$ is an equivalence.
        \end{itemize}
\end{exercise}
\begin{answer}
    \begin{itemize}
        \item[(a)]
            Since $F$ is an equivalance there is a functor $G\from \cat{B} \to \cat{A}$ and a natural isomorphism $\eta\from 1_{\cat{A}} \to G \circ F$.
            So the following diagram commutes for all $f\from A \to A'$:
            \[
                \begin{tikzcd}
                    A \arrow{r}{f} \arrow{d}{\eta_A} & A' \arrow{d}{\eta_{A'}} \\
                    G(F(A)) \arrow{r}{G(F(f))}       & G(F(A'))
                \end{tikzcd}
            \]
            Also we have a natural isomorphism $\epsln\from F \circ G \to 1_{\cat{B}}$ such that for all $g\from B \to B'$ the diagram commutes:
            \[
                \begin{tikzcd}
                    F(G(B)) \arrow{d}{\epsln_{B}} \arrow{r}{F(G(g))} & F(G(B')) \arrow{d}{\epsln_{B'}} \\
                    B \arrow{r}{g}  & B'
                \end{tikzcd}
            \]
            \begin{description}
                \item[$F$ is faithful:]
                    Let $A, A' \in \cat{A}$ and $f,f'\from A \to A'$ such that $F(f) = F(f')$.
                    Using the naturality axiom for $\eta$ we obtain our goal.
                    \begin{align*}
                        f & = \eta_{A'}^{-1} \circ \eta_{A'} \circ f\\
                          & = \eta_{A'}^{-1} \circ G(F(f)) \circ \eta_A\\
                          & = \eta_{A'}^{-1} \circ G(F(f')) \circ \eta_A\\
                          & = \eta_{A'}^{-1} \circ \eta_{A'} \circ f'\\
                          & = f'
                    \end{align*}
                \item[$F$ is full:]
                    %TODO
                    Let $A, A' \in \cat{A}$ and $g \from F(A) \to F(A')$.
                \item[$F$ is essentially surjective on objects:]
                    Let $B \in \cat{B}$.
                    Then $F(G(B)) \iso B$ by $\epsln_B$.
            \end{description}
        \item[(b)]
            For each $B \in \cat{B}$ the class of objects $A \in \cat{A}$ which fulfill $F(A) \iso B$ is non empty, because $F$ is essentially surjective on objects.
            We define $G(B)$ as such an $A$ using the axiom of choice.
            In addition we have an isomorphism $\epsln_B\from F(G(B)) \to B$.
            Let $g\from B \to B'$ be a morphism in $\cat{B}$.
            Then $g' := \epsln_{B'}^{-1} \circ g \circ \epsln_{B} \from F(G(B)) \to F(G(B'))$ is a morphism in the image of $F$.
            We extend $G$ to a functor by mapping each morphism $g\from B \to B'$ to the morphism $f\from G(B) \to G(B')$ such that $g' = F(f)$.
            This morphism $f$ exists and is unique because $F$ is full and faithful.
            We have to show that $\epsln$ is a natural transformation.
            \[
                \begin{tikzcd}
                    F(G(B)) \arrow{d}{\epsln_{B}} \arrow{r}{F(G(g))} & F(G(B')) \arrow{d}{\epsln_{B'}} \\
                    B \arrow{r}{g}                                   & B'
                \end{tikzcd}
            \]
            The diagram commutes since $F(G(g)) = g' = \epsln_{B'}^{-1} \circ g \circ \epsln_{B}$.\\
            Next we define $\eta_A$ such that $F(\eta_A) = \epsln_{F(A)}^{-1}$.
            Again this is possible because $\epsln_{F(A)}^{-1}\from F(A) \to F(G(F(A)))$ is in the image of $F$.
            \[
                \begin{tikzcd}
                    A \arrow{d}{\eta_A} \arrow{r}{f} & A' \arrow{d}{\eta_{A'}} \\
                    G(F(A)) \arrow{r}{G(F(f))}       & G(F(A'))
                \end{tikzcd}
            \]
            \begin{align*}
                F(G(F(f))) & = \epsln_{F(A')}^{-1} \circ F(f) \circ \epsln_{F(A)}\\
                           & = F(\eta_{A'}) \circ F(f) \circ F(\eta_A^{-1})\\
                           & = F(\eta_{A'} \circ f \circ \eta_A^{-1})
            \end{align*}
            We obtain $G(F(f)) = \eta_{A'} \circ f \circ \eta_A^{-1}$ because $F$ is faithful.
            This shows that $\eta$ is a natural isomorphism.
            Therefore $F$ is an equivalence.
    \end{itemize}
\end{answer}

\begin{exercise}
This exercise makes precise the idea that linear algebra can equivalently
be done with matrices or with linear maps.

Fix a field $K$.  Let $\Mt$ be the category whose objects are the natural
numbers and with
\[ \Mt(m, n) = \{ n \times m \text{ matrices over } K \}. \]
Prove that $\Mt$ is equivalent to $\FDVect_K$, the category of
finite-dimensional vector spaces over $K$.
Does your equivalence involve a \emph{canonical} functor from $\Mt$ to $\FDVect_K$, or from $\FDVect_K$ to $\Mt$?
\end{exercise}
\begin{answer}
    Define a functor $F\from \Mt \to \FDVect_K$:
    \begin{align*}
        F(n) & := K^n && n \in \NN\\
        F(A\from m \to n) & := \lambda\, x. \, A \cdot x && A \text{ is a } n \times m \text{ matrix}
    \end{align*}
    Note that $F$ is indeed a functor, since matrix multiplication is associative.
    \begin{description}
        \item[$F$ is faithful:]
            Let $A, B\from m \to n$, that is $n \times m$ matrices over $K$, such that $F(A) = F(B)$.
            By using the standard basis $(e_j)_{j = 1}^m$ in $K^m$ we get:
            \[A \cdot e_j = F(A)(e_j) = F(B)(e_j) = B \cdot e_j \quad\forall 1 \le j \le m\]
            So each column of $A$ and $B$ are the same.
            Thus $A = B$.
        \item[$F$ is full:]
            Let $f\from K^m \to K^n$ be a linear function.
            We define the $n \times m$ matrix $A$ columnwise.
            The $j$-th column of $A$ is $f(e_j)$, where $e_j$ is the $j$-th standard basis vector.
            By this definition $F(A)$ and $f$ conincide on the standard basis.
            Linear algebra tells us that in this case $F(A)$ and $f$ coincide on all vectors.
            So $A$ is mapped to $f$ under $F$.
        \item[$F$ is essentially surjective on objects:]
            Let $V$ be a finite-dimensional $K$-vectorspace.
            Then $V$ has some dimension $n \in \NN$.
            Thus $V \iso K^n = F(n)$.
    \end{description}
    Thus $F$ is an equivalence and so $\Mt \eqv \FDVect_K$.
    The functor $F$ is even canonical since we didn't make arbitrary choices.
    But the functor from $\FDVect_K$ to $\Mt$ involves choosing a basis like in the case for showing that $F$ is full.
\end{answer}

\end{document}

%%% Local Variables:
%%% mode: latex
%%% TeX-master: t
%%% End:
