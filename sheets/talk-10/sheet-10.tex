\def\pathToRoot{../../}\documentclass{article}

\usepackage{nag}
\makeatletter
\@ifclassloaded{beamer}{}
{\usepackage[small,compact]{titlesec}}
\makeatother
\usepackage[utf8]{inputenc}
\usepackage[T1]{fontenc}
\usepackage{lmodern}
\usepackage{color}
\usepackage{parskip}
\usepackage{needspace}
\usepackage{microtype}
\usepackage{mathtools}
\usepackage{xifthen}
\usepackage{xpatch}
\usepackage{enumitem}
\usepackage{mdwlist}
\usepackage{bussproofs}
\EnableBpAbbreviations
\usepackage{tabu}
\usepackage{amssymb}
\usepackage{amsmath}
\usepackage{amsthm}
%grober hack, der den groben hack von parskip bei den amsthm sachen korrigiert
\begingroup
    \makeatletter
       \@for\theoremstyle:=definition,remark,plain\do{%
            \expandafter\g@addto@macro\csname th@\theoremstyle\endcsname{%
                        \addtolength\thm@preskip\parskip
             }%
        }
\endgroup
\usepackage[UKenglish]{babel}
\usepackage{xparse}
\usepackage{adjustbox}
\usepackage{geometry}
\usepackage{booktabs}
\usepackage{multicol}
\usepackage{soul}
\usepackage{calc}
\usepackage{textcase}
\usepackage{stmaryrd}
\usepackage{marvosym}
\usepackage{wasysym}
\usepackage{pifont}
\newcommand{\cmark}{\ding{51}}
\newcommand{\xmark}{\ding{55}}
\usepackage{tikz}
\usetikzlibrary{trees, backgrounds, shapes, chains, decorations.text, decorations.pathreplacing, circuits.logic.IEC, patterns, matrix}
\usepackage{tikz-qtree}
\usepackage{tikzsymbols}
\usepackage{fancyvrb}
\usepackage{fancyhdr}
\usepackage{verbatim}
\usepackage[framemethod=tikz]{mdframed}
\usepackage{lastpage}
\usepackage{pgfpages}
\usepackage{csquotes}
\usepackage{longtable}
\usepackage{ragged2e}
%\usepackage{stackengine}
\usepackage{censor}
\usepackage{expl3}
\usepackage{multirow}
\usepackage{hyperref}
\usepackage{environ}


% Package for Cateogry diagrams:

\usepackage{tikz-cd}



\ifcsdef{labelenumi}{
\renewcommand{\labelenumi}{(\alph{enumi})}
\renewcommand{\labelenumii}{(\roman{enumii})}
}{}

\input{\pathToRoot headers/definitions}



\tikzset{
    normal/.style={draw, semithick},
    n/.style={style=normal, circle, inner sep=1mm, minimum size=8mm},
    l/.style={style=normal, rounded corners=1mm, inner sep=1mm, minimum size=6mm},
    e/.style={style=normal, shorten >=1mm, shorten <=1mm, ->, >=stealth},
    syntax/.style={style=normal, ellipse, minimum height=6mm, minimum width=8mm}, % nodes in syntax trees
    inner/.style={style=normal, minimum size=4mm}, % inner leaves or root in normal trees
    leaf/.style={style=normal, circle, minimum size=4mm}, % leaves in normal trees
    te/.style={style=normal}, % edges in a tree
    be/.style={style=e, dashed} % binding edge
}

\newcommand{\syntaxtree}[1]{ % DEPRECATED - use tikzsyntaxtree
    \begin{tikzpicture}[baseline=(current bounding box.north)]
        \tikzset{grow=down}
        \tikzset{every node/.style={syntax}}
        \tikzset{edge from parent/.style=
            {te,
                edge from parent path={(\tikzparentnode) -- (\tikzchildnode)}}}
        \Tree #1
    \end{tikzpicture}
}

\newenvironment{tikzsyntaxtree}[1][]{
    \begin{tikzpicture}[baseline=(current bounding box.north), #1]
    \tikzset{grow=down}
    \tikzset{every tree node/.style={syntax}}
    \tikzset{edge from parent/.style={te, edge from parent path={(\tikzparentnode) -- (\tikzchildnode)}}}
}{
    \end{tikzpicture}
}


\newcommand{\DisplayScaledProof}{\maxsizebox{\linewidth}{!}{\DisplayProof}}
\newcommand{\DisplayTopProof}{\adjustbox{valign=t}{\DisplayProof}}
\newcommand{\DisplayScaledTopProof}{\adjustbox{valign=t}{\maxsizebox{\linewidth}{!}{\DisplayProof}}}


\newcolumntype{P}[1]{>{\RaggedRight\hspace{0pt}}p{#1}}

\newenvironment{prooftable}
{
    \begin{longtable}{>{\footnotesize}p{0.33\textwidth}>{\footnotesize}p{0.33\textwidth}|>{\footnotesize}P{0.15\textwidth}}
    \normalsize Textbeweis & \normalsize Erklärungen & \normalsize Schlussregel\\\hline
    \endhead
}
{
    \end{longtable}
}


\theoremstyle{definition}
\newtheorem*{definition*}{Definition} % Definition ohne Nummer
\newtheorem*{inferenceRule*}{Schlussregel}

\usepackage{titling}
\geometry{a4paper,left=2cm,right=2cm,top=2cm,bottom=3cm}


\newcommand{\licenseccjuliachristian}{\def\islicenseccjuliachristian{}}
\newcommand{\suppresslicense}{\def\issuppresslicense{}}


\AtBeginDocument{
    \pagestyle{fancy}
    \renewcommand{\headrulewidth}{0pt}
    \renewcommand{\footrulewidth}{1pt}
    \fancyhead{}
    \fancyfoot[C]{\thepage~/~\pageref{LastPage}}
    \fancyfoot[R]{\footnotesize exercise sheet from \\ \theauthor}

}


\newcommand{\pgbreakhere}{\Needspace*{4\baselineskip}}
\newcommand{\pgbreakHere}{\Needspace*{10\baselineskip}}
\newcommand{\pgbreakHERE}{\Needspace*{15\baselineskip}}

\newcommand{\raisedrule}[2][0em]{\leavevmode\leaders\hbox{\rule[#1]{1pt}{#2}}\hfill\kern0pt}

% inspired by http://tex.stackexchange.com/questions/242294/suppress-parskip-only-after-a-specific-paragraph
\makeatletter
\newlength\noparskip@parskip % used to store a backup of the parskip value
\newboolean{noparskip@triggered} % flag to indicate that noparskip was run in the current paragraph
\setboolean{noparskip@triggered}{false}
\newboolean{noparskip@active} % flag to indicate that parskip should be restored after this paragraph
\setboolean{noparskip@active}{false}
\let\noparskip@par\par % store a backup of the \par command
\@setpar{% redefine \par with the means of ltpar.dtx to stay compatible to enumerate and itemize
    \ifhmode% since we're counting occurrences of \par, \par\par would be a problem, so check that we are actually ending a paragraph
        \ifthenelse{\boolean{noparskip@active}}{%
            \setlength\parskip\noparskip@parskip% restore parskip
            \setboolean{noparskip@active}{false}% remember not the restore parskip again
        }{}%
        \ifthenelse{\boolean{noparskip@triggered}}{%
            \ifthenelse{\boolean{noparskip@active}}{}{
                % we are triggering noparskip and not currently in a noparskip already
                \setlength\noparskip@parskip\parskip % copy the current parskip into the backup variable
            }%
            \setboolean{noparskip@triggered}{false}% paragraph is ending, so noparskip is no longer triggered
            \setlength\parskip{0pt}% no parskip when the next paragraph begins
            \setboolean{noparskip@active}{true}% parskip must be restored by the next par
        }{}%
    \fi%
    \noparskip@par% run the original par command
}
\def\noparskip@backout{%
    \ifthenelse{\boolean{noparskip@active}}{%
        % a list is beginning and parskip is currently set to zero, wich would mess up the list
        \setlength\parskip{\noparskip@parskip}% restore parskip before the list begins
        \setboolean{noparskip@active}{false}%
    }{}%
    \setboolean{noparskip@triggered}{false}% there's no sense in keeping noparskip triggered throughout a list
}
\xpretocmd\begin{%
    \ifstrequal{#1}{enumerate}{\noparskip@backout}{}%
    \ifstrequal{#1}{itemize}{\noparskip@backout}{}%
    \ifstrequal{#1}{list}{\noparskip@backout}{}%
    \ifstrequal{#1}{proof}{\noparskip@backout}{}%
}{}{}
\def\noparskip{%
    \leavevmode% ensure that we are within a paragraph
    \setboolean{noparskip@triggered}{true}% trigger noparskip
}
\makeatother

\newcommand{\noparskipworkaround}{} % DEPRECATED and no longer needed


\newcommand{\head}[1]{
    {
        \setlength{\parskip}{0pt}
        \hrule height 1pt
        \vspace{.2cm}
        Saarland University \hfill Category Theory Seminar 2017\par
        Programming Systems Lab \hfill \small\url{https://courses.ps.uni-saarland.de/ct_ss17/}\par
        \tiny\raisedrule[0mm]{1pt}
        \vspace{2ex}
        \begin{center}
            \Large
            \textbf{#1}\par
            \raisedrule[2mm]{1pt}
        \end{center}
        \vspace{3ex}
    }
}

\newenvironment{leftframedparagraph}{\begin{mdframed}[hidealllines = true, leftline = true, innerleftmargin = 2ex, innerrightmargin = 0pt,
innertopmargin = 0pt, innerbottommargin = 2pt, skipabove=2ex, skipbelow=1ex, outerlinewidth = 0ex, innerlinewidth = 0.5ex]}{\end{mdframed}}
\newenvironment{leftframed}{\begin{mdframed}[hidealllines = true, leftline = true, innerleftmargin = 2ex, innerrightmargin = 0pt,
innertopmargin = 0pt, innerbottommargin = 0pt, skipabove=2ex, skipbelow=1ex, outerlinewidth = 0ex, innerlinewidth = 0.5ex]}{\end{mdframed}}

%%% Local Variables:
%%% mode: latex
%%% TeX-master: t
%%% End:


\newcommand{\uebunghead}[3][Exercise sheet:]{\def\sheetid{#2}\head{#1 #2\ifthenelse{\isundefined{\issolution}}{}{ \ifthenelse{\isundefined{\ismarking}}{(Possible solutions)}{(Marking)}} \\ #3}}

\licenseccjuliachristian


\newcommand{\amountofpoints}[1]{\ifstrequal{#1}{1}{1~Punkt}{#1~Punkte}}


% marking implies solution
\ifthenelse{\isundefined{\ismarking}}{}{\def\issolution{}}


%%%Environments
\newcounter{ExamExerciseCounter} % will only be used in exams, but must be defined here so ExerciseCounter can be reset when ExamExericise counts
\setcounter{ExamExerciseCounter}{0}
\newcounter{ExerciseCounter}[ExamExerciseCounter]
\setcounter{ExerciseCounter}{0}

\newcommand{\ExerciseNumber}{\sheetid.\arabic{ExerciseCounter}}
\renewcommand{\theExerciseCounter}{\ExerciseNumber}

\newcommand{\ExercisePointHook}[1]{}

%Aufgaben-Umgebung
\NewDocumentEnvironment{exercise}{od<>}{
    \refstepcounter{ExerciseCounter}
    \pgbreakhere
    \vspace{1ex}\textbf{Exercise\ \ExerciseNumber}%
    \IfNoValueF{#1}{ \emph{(#1)}}%
    \IfNoValueF{#2}{\hfill(\amountofpoints{#2})}%
    \IfNoValueF{#2}{\ExercisePointHook{#2}}%
    \noparskip\par\nopagebreak
}{
    \par
    \vspace{2ex}
}

\newcommand{\exercisesOnly}[1]{\ifthenelse{\isundefined{\issolution}}{#1}{}}

%Loesungs-Umgebung
\newenvironment{answer}
{
    \ifthenelse{\isundefined{\issolution}}
    {
        \comment
    }{
        \vspace{1ex}\textsl{Sample solution \ExerciseNumber}\noparskip\par\nopagebreak
    }
}{
    \ifthenelse{\isundefined{\issolution}}
    {
    }{
        \vspace{1ex}
        \hspace*{\fill}
    }
}

\newenvironment{marking}
{%
    \ifthenelse{\isundefined{\ismarking}}%
    {%
        \comment%
    }{%
        \color{red}
    }%
}{%
    \ifthenelse{\isundefined{\ismarking}}%
    {%
    }{%
    }%
}

\newenvironment{example}{\begin{leftframedparagraph}\paragraph{Example:}}{\end{leftframedparagraph}}
\newenvironment{hint}{\paragraph{Hint:}}{}
\newenvironment{caution}{\paragraph{Caution:}}{}
\newenvironment{definition}[1]{\begin{leftframedparagraph}\paragraph{Definition (#1):}}{\end{leftframedparagraph}}


\begin{document}

% Use Basis x or Talk x, where x is the number of the session
\uebunghead{Talk 10}{Presheaves, Representables and the Yoneda Lemma}
\author{Dominik Wagner}

\begin{hint}
  Read chapter 4 in Leinster and chapter 8 in Awodey.
\end{hint}

\begin{definition}{Right $M$-Sets}
  Let $M$ be a monoid. 
  \begin{itemize}
  \item A \emph{right $M$-set} is a set $S$ together with a function
    \begin{align*}
      \cdot\from S\times M&\to S\\
      (s,m)&\mapsto s\cdot m
    \end{align*}
    such that $s\cdot (mm')=(s\cdot m)\cdot m'$ and $s\cdot 1_M=s$ for all $m,m'\in M$ and $s\in S$.
    \item A homomorphism between two right $M$-sets $S_1$ and $S_2$ is a mapping $\alpha\from S_1\to S_2$ satisfying $\alpha(s\cdot_{S_1} m)=\alpha(s)\cdot_{S_2}m$.
    \item The \emph{right regular representation} $\underline{M}$ is the $M$-set $M$ together with the right-operation $s\cdot m=sm$, i.e. the underlying monoid-operation.
  \end{itemize}

\end{definition}

\begin{exercise}[Not difficult despite its length]
  In this exercise we prove the Yoneda Lemma for the special case of one-object categories, i.e. monoids regarded as categories.
  We exploit the correspondence between presheaves on one-object categories (monoids $M$) and right $M$-sets.

  \begin{enumerate}
  \item Make yourself familiar again with this correspondence (e.g. by reading the examples 1.2.14 and 1.2.8 in Leinster).
  \item Let $M$ be a monoid. Show that the $M$-set corresponding to the unique(!) representable functor $M^\op\to \Set$ is the right regular representation $\underline M$.
  \item Now let $S$ be any right $M$-set. Show that for each $s\in S$ there is a unique homomorphism $\alpha\from\underline M\to S$ of $M$-sets such that $\alpha(1)=s$. Deduce that 
    \begin{align*}
      \{\text{homomorphisms }\underline M\to S\}\iso S
    \end{align*}
  \item Deduce that
    \begin{align*}
      [M^\op,\Set](H_A, F)\iso F(A)
    \end{align*}
    holds for all presheaves $F\from M^\op\to\Set$.
  \end{enumerate}
\end{exercise}

\begin{answer}
  \begin{enumerate}
  \item Done!
  \item Let $A$ be the unique object corresponding to $M$. In what follows we identify elements $m$ of the monoid $M$ with arrows $A\xrightarrow{m}A$. Then, $H_A(A)=\{\text{maps } A\to A\}= M$. Furthermore, for all $m,m'\in M$, $m\cdot m'=H_A(m')(m)=m\of m'=mm'$. Hence, $H_A$ contains exactly the same information as $\underline M$.
  \item Let $S$ be a right $M$-set. Assume there is a homomorphism $\alpha\from \underline M\to S$. Then for each $s\in S$, 
    \begin{align*}
      \label{eq:defa}
      \alpha(s)=\alpha(1\cdot_M s)=\alpha(1)\cdot_S s.
    \end{align*}
    Thus, $\alpha$ is completely defined by its action on $1$ if it exists. At the same time, this equation completely defines a homomorphism since
    \begin{align*}
      \alpha(s)\cdot m=(\alpha(1)\cdot s)\cdot m=\alpha(1)\cdot(s\cdot m)=\alpha(s\cdot m)
    \end{align*}
    as required.
    Therefore, the function
    \begin{align*}
      \{\text{homomorphisms }\underline M\to S\}&\to S\\
      \alpha&\mapsto \alpha(1)
    \end{align*}
    is an isomorphism.
  \item To conclude the Yoneda lemma we still have to show that natural transformations $F_1\to F_2$ are the same as homomorphisms between $S_1$ and $S_2$ where $S_1,S_2$ are the $M$-sets corresponding to the presheaves $F_1$ and $F_2$. 

    To see this let us instantiate the definition of naturality for our special case. It states that for each arrow $A\xrightarrow{m}A$ the following diagram commutes:

    \begin{align*}
      \xymatrix{
      F_1(A)=S_1 \ar[r]^{F_1(m)=-\cdot m} \ar[d]^{\alpha} & F_1(A)=S_1 \ar[d]_{\alpha} \\
      F_2(A)=S_2 \ar[r]^{F_2(m)=-\cdot m} & F_2(A)=S_2
                   }
    \end{align*}
    (since there is only one object) or more concretely for each $s\in S_1$,
    \begin{align*}
      \alpha(s\cdot m)=\alpha(s)\cdot m.
    \end{align*}

    Now the Yoneda lemma (for our special case) is an easy consequence:
    \begin{align*}
      [M^\op,\Set](H_A, F)\iso \{\text{homomorphisms }\underline M\to S\}\iso S= F(A)
    \end{align*}
    where $S$ is the $M$-Set corresponding to $F$.
  \end{enumerate}
\end{answer}

\begin{exercise}
  Recall the following re-formulation of the Yoneda principle: For any locally small category $\cat A$ and objects $A,A'\in\cat A$:
  \begin{align*}
    \cat A(B,A)\iso \cat  A(B,A')\text{ naturally in } B\in\cat A
  \end{align*}
implies $A\iso A'$. Show that naturality is really necessary.
\hint{There is a counterexample with two objects.}
\end{exercise}
\begin{answer}
consider the following category $\cat A$: 
\begin{align*}
  \xymatrix{
  A \ar@(ul,ur)^{1_A,f} \ar@/^0.3pc/@[][r]^{f_1,f_2} & B \ar@/^0.3pc/@[][l]^{g_1,g_2} \ar@(ul,ur)^{1_B,g}
                                                       } 
  \end{align*}
  i.e. between any pair of object there are two arrows, where the composition of $f_i$ and $g_j$ ($i,j\in\{1,2\}$) is given by
  \begin{align*}
    f_i\of g_j&=g
  \end{align*}
  and the remaining compositions can be chosen arbitrarily.
  
  Obviously,
  \begin{align*}
    \cat A(C,A)\iso\cat A(C,B)
  \end{align*}
  for all $C\in\cat A$ because all homsets contain exactly two elements. However $A\not\iso B$ because by construction neither $g_1$ nor $g_2$ (which are the only maps between $B$ and $A$) have a left inverse.
\end{answer}

\begin{exercise}
  Let $\cat A$ be a locally small CCC. Use the Yoneda embedding to show that for any objects $A,B,C\in\cat A$,
  \begin{align*}
    (A\times B)^C\iso A^C\times B^C
  \end{align*}
  If $\cat A$ also has binary coproducts, show that also
  \begin{align*}
    A^{B+C}&\iso A^B\times A^C
  \end{align*}
\end{exercise}

\begin{answer}
  \begin{itemize}
  \item It holds
    \begin{align*}
      \cat A(X,(A\times B)^C&\iso\cat A(X\times C,A\times B)&\cat A\text{ cartesian closed}\\
                            &\iso\cat A(X\times C,A)\times \cat A(X\times C,B)&\text{dual of \ref{ex:det}-\ref{ex:col}}\\
                            &\iso\cat A(X,A^C)\times \cat A(X,B^C)&\cat A\text{ cartesian closed}\\
                            &\iso\cat A(X,A^C\times B^C)&\text{dual of \ref{ex:det}-\ref{ex:col}}
    \end{align*}
    naturally in $X\in\cat A$. Hence, $(A\times B)^C\iso A^C\times B^C$.

    To see that the first isomorphsim is natural in $X$, suppose an arrow $f\from Y\to X$. Naturality means that for any $g\from X\to (A\times B)^C$
    \begin{align*}
      \overline{g\of f}=\epsilon\of ((g\of f)\times 1_{A\times B})=\epsilon\of ((g\times 1_{A\times B})\of(f\times 1_{A\times B})=\overline g\of (f\times 1).
    \end{align*}
    The other isomorphisms can be equally simply shown to be natural in $X$.
  \item It holds
    \begin{align*}
      \cat A(X,A^{B+C}&\iso\cat A(X\times (B+C),A)&\cat A\text{ cartesian closed}\\
                      &\iso\cat A((X\times B)+(X\times C),A)&\text{``distributivity''-lemma}\\
                      &\iso\cat A(X\times B,A)\times\cat A(X\times C, A)&\text{\ref{ex:det}-\ref{ex:col}}\\
                      &\iso\cat A(X,A^B)\times\cat A(X, A^C)&\cat A\text{ cartesian closed}\\
                      &\iso\cat A(X,A^B\times A^C)&\text{dual of \ref{ex:det}-\ref{ex:col}}
    \end{align*}
    naturally in $X\in\cat A$. Hence, $A^{B+C}\iso A^B\times A^C$.
  \end{itemize}
\end{answer}

\begin{exercise}
  Let $\cat B$ be a category and $J\from\cat C\to\cat D$ a functor. There is an induced functor
  \begin{align*}
    J\of -\from[\cat B,\cat C]\to [\cat B,\cat D]
  \end{align*}
  defined by composition with $J$.
  \begin{enumerate}
  \item Show that if $J$ is full and faithful then so is $J\of -$.
  \item Deduce that if $J$ is full and faithful and $G,G'\from\cat B\to\cat C$ with $J\of G\iso J\of G'$ then $G\iso G'$.
  \item Now deduce that right adjoints are unique: if $F\from \cat A\to\cat B$ and $G,G'\from\cat B\to \cat A$ with $F\dashv G$ and $F\dashv G'$ then $G\iso G'$. 
    \hint{The Yoneda embedding is full and faithful.}
  \end{enumerate}
\end{exercise}
\begin{answer}
  \begin{enumerate}
  \item Clearly, $J\of-$ is defined on arrows $\alpha\from F\to G$ for $F,G\from\cat B\to\cat C$ by
    \begin{align*}
      J\of -(\alpha)&=\alpha'
    \end{align*} 
    where $\alpha$ is a natural transformation $J\of F\to J\of G$ defined by
    \begin{align*}
      \alpha'_B=J(\alpha_B)\from J\of F(B)\to J\of G(B)
    \end{align*}
    for $B\in\cat B$.

    Full and faithfulness of $J$ means that for each $C,C'\in\cat C$ the function
    \begin{align*}
      J_{C,C'}&\from\cat C(C,C)\to \cat D(J(C),J(C'))\\
      f&\mapsto J(f)
    \end{align*}
    is bijective.

    To show that $J\of-$ is full and faithful, let $F,G\from\cat B\to\cat C$ be natural transformations and $\beta\from J\of F\to J\of G$ be a natural transformation.
    Define a natural transformation $\alpha\from F\to G$ by 
    \begin{align*}
      \alpha_B=J_{F(B),G(B)}^{-1}(\beta_B)
    \end{align*}
    for $B\in\cat B$. This is well-defined by full- and faithfulness and defines indeed a natural transformation.

    Clearly, $J\of -(\alpha)=\beta$. This shows fullness of $J\of -$. For faithfulness assume a natural transformation $\alpha'\from F\to G$ and suppose 
    \begin{align*}
      J\of-(\alpha)=J\of-(\alpha')
    \end{align*}
    By definition this means
    \begin{align*}
      J(\alpha_B)=J(\alpha'_B)
    \end{align*}
    for any $B\in\cat B$. Faithfulness of $J$ implies $\alpha_B=\alpha'_B$ for any $B\in\cat B$, i.e. $\alpha=\alpha'$.
  \item \label{ex:lem} $J\of -\from[\cat B,\cat C]\to [\cat B,\cat D]$ is full and faithful by the previous part. Hence for $G,G'\in [\cat B,\cat C]$, $G\iso G'$ if and only if $J\of G=J\of-(G)\iso J\of-(G')=J\of G'$.
  \item By adjointness we get
    \begin{align*}
      H_{G(B)}(A)=\cat A(A,G(B))\iso\cat B(F(A),B)\iso\cat A(A,G'(B))=H_{G'(B)}(A)
    \end{align*}
    naturally in $A\in\cat A$ and $B\in\cat B$. In particular,
    \begin{align*}
      (H_-\of G)(B)=H_{G(B)}\iso H_{G'(B)}=(H_-\of G')(B)
    \end{align*}
    naturally in $B$, i.e. $H_-\of G\iso H_-\of G'$. Due to the fact that the Yoneda embedding $H_-$ is full and faithful and part \ref{ex:lem}, $G\iso G'$.

    % By adjointness we get
%     \begin{align*}
%       \cat A(A,G(B))\iso\cat B(F(A),B)\iso\cat A(A,G'(B))
%     \end{align*}
%     naturally in $A\in\cat A$ and $B\in\cat B$. % This means 
%     % \begin{align*}
%     %   H_{G(B)}(A)\iso H_{G'(B)}(A)
%     % \end{align*}
%     % naturally in $A$, i.e. $(H_-\of G)(B)=H_{G(B)}\iso H_{G'(B)}=(H_-\of G')(B)$.
% This means there is a natural transformation $\eta_B$ that is an isomorphism and for which
%     \begin{align*}
%                 H_{G(B)}(A)\xrightarrow{\eta_{B,A}}H_{G'(B)}(A)
%     \end{align*}
%     naturally in $A$ and $B$.
    
%     As a consequence $H_-(G(B))=H_{G(B)}\iso H_{G'(B)}=H_-(G'(B))$ and the isomorphism is given by $\eta_B$. By the Yoneda principle, $G(B)\iso G'(B)$. 

%     This isomorphism is even natural in $B$: Since $\cat A(A,G(B))\iso\cat A(A,G'(B))$ naturally in $B$ the following commutes for any $f\from B\to B'$:
%     \begin{align*}
%       \xymatrix{
%       H_{G(B)} \ar[r]^{H_{G(f)}}\ar[d]^{\eta_B} & H_{G(B')} \ar[d]_{\eta_{B'}}\\
%       H_{G(B')} \ar[r]^{H_{G'(f)}} & H_{G'(B')} 
%                                         } 
%   \end{align*}
%   Because the Yoneda embedding $H_-$ is full and faithful this means
%   \begin{align*}
%     \xymatrix{
%     G(B) \ar[r]^{G(f)}\ar[d]^{\zeta_B} & G(B') \ar[d]_{\zeta_{B'}}\\
%     G(B') \ar[r]^{G'(f)} & G'(B') 
%                                    } 
%   \end{align*}
%   commutes where $\zeta_B$ and $\zeta_{B'}$ are the isomorphism corresponding to $\eta_B$ and $\eta_{B'}$ respectively under $H_-$. Another consequence is that $G(B)\iso G'(B)$ naturally in $B$, i.e. $$.
  \end{enumerate}
\end{answer}

\begin{exercise}
  Let $\cat A$ be a small category. Prove that the representable functors \emph{generate} the presheaf category $[\cat A^\op,\Set]$ in the folowing sense: Given any objects $F,G\in[\cat A^\op,\Set]$ and natural transformations $\alpha,\beta\from F\to G$, if for every representable functor $H_A$ and natural transformation $\gamma\from H_A\to F$, one has $\alpha\of\gamma=\beta\of\gamma$, then $\alpha=\beta$. Thus, the arrows in $[\cat A^\op,\Set]$ are determined by their effect on generalized elements based at representables.
\end{exercise}

\begin{answer}
  Let $F,G,\alpha,\beta$ be given as stated above. By the naturality statement in the Yoneda lemma we have that
  \begin{align*}
    \xymatrix{
    [\cat A^\op,\Set](H_A,F) \ar[r]^{\alpha\of -} & [\cat A^\op,\Set](H_A,G) \\
    F(A) \ar[r]^{\alpha_A} \ar[u]^{\iso} & G(A) \ar[u]^{\iso}
                             } 
  \end{align*}
commutes for any $A\in\cat A$. In particular, with the definitions from the proof of the Yoneda lemma we have that 
\begin{align*}
    \xymatrix{
    [\cat A^\op,\Set](H_A,F) \ar[r]^{\alpha\of -} & [\cat A^\op,\Set](H_A,G) \ar[d]^{\widehat{-}} \\
    F(A) \ar[r]^{\alpha_A} \ar[u]^{\widetilde{-}} & G(A) 
                             } 
\end{align*}
commutes for any $A\in\cat A$. Consequently, for any $A\in\cat A$ and $a\in F(A)$,
\begin{align*}
  \alpha_A(a)=\widehat{\alpha\of\widetilde a}.
\end{align*}
Analogously,
\begin{align*}
  \beta_A(a)=\widehat{\beta\of\widetilde a}.
\end{align*}
By assumption ($\widetilde a\from H_A\to F$)
\begin{align*}
  \alpha\of\widetilde a=\beta\of\widetilde a
\end{align*}
holds and therefore
\begin{align*}
  \alpha_A(a)=\beta_A(a)
\end{align*}
for any $A\in\cat A$ and $a\in F(A)$, which means $\alpha=\beta$.
\end{answer}

% \begin{definition}{Preservation of Limits}
%   Let $\mathbf I$ be a small category. A functor $F\from \cat A\to\cat B$ \emph{preserves colimits of shape $\mathbf I$} if for all diagrams $D\from\mathbf I\to\cat A$ and all cocones $(D(I)\xrightarrow{p_I}A)_{I\in \mathbf I}$ on $D$,
%   \begin{align*}
%     &(D(I)\xrightarrow{p_I}A)_{I\in \mathbf I} \text{ is a limit cone on $D$ in } \cat A\\
%     \Longrightarrow&(F\of D(I)\xrightarrow{F(p_I)}F(A))_{I\in \mathbf I} \text{ is a limit cone on $F\of D$ in } \cat A
%   \end{align*}

%   A functor $F\from \cat A\to\cat B$ \emph{preserves limits} if it preserves limits of shape $\mathbf I$ for all small categories $\mathbf I$.
% \end{definition}
\begin{exercise}
  \label{ex:det}
  Fill in some of the missing details in the proof that for any small category $\cat A$ the presheaf category $[A^\op,\Set]$ is cartesian closed:
  \begin{enumerate}
  \item \label{ex:col}Show that for any $A\in\cat A$, $H_A$ maps all colimits to limits such that $$H_A(\lim_{\to\mathbf I} D)\iso \lim_{\leftarrow\mathbf I}(H_A\of D)$$ for all small categories $\mathbf I$ and diagrams $D\from \mathbf I\to\cat A$. What is the dual statement?
    \hint{It suffices to show that this holds for all coproducts and coequalizers.}
  \item Let $F\from \cat A\to\Set$ be a presheaf. Show that the functor $-\times F\from [\cat A^\op,\Set]\to[\cat A^\op,\Set]$ preserves colimits. That is: For any small category $\mathbf I$ and functor $D\from\mathbf I\to [\cat A^\op,\Set]$,
    \begin{align*}
      \lim_{\to \mathbf I} (D\times F)\iso (\lim_{\to \mathbf I} D)\times F.
    \end{align*}
    You may assume that $[\cat A^\op,\Set]$ has all limits. (In particular limits, which are functors $\cat A^\op\to\Set$, are taken ``pointwise'': $$\left(\lim_{\to I\in\mathbf I} D(I)\right)(A) \iso \lim_{\to I\in\mathbf I} \left(D(I)(A)\right)$$ for all shapes and diagrams.)
    \hint{Use the fact that $[\cat A^\op,\Set]$ has all limits to deduce that there is an arrow $\lim_{\to \mathbf I} (D\times F)\to (\lim_{\to \mathbf I} D)\times F$ and use the Yoneda principle to show that this is an isomorphism.}
  \end{enumerate}
\end{exercise}

\begin{answer}
  \begin{enumerate}
  \item
    By the hint it suffices that coproducts and coequalizers are mapped to the corresponding products and equalizers, respectively (because each colimit can be obtained from coproducts and coequalizers).
    \begin{itemize}
    \item Coproducts are mapped to products: Let $(B_i)_{i\in I}$ be an arbitrary family of objects and suppose $B=\sum\limits_{i\in I}B_i$ exists. Then for any object $A$ and family of maps $(f_i\from B_i\to A)_{i\in I}$ there is a unique $\bar f\from B\to A$ s.t. $f_i=\bar f\of p_i$ for all $i\in I$ because of the universality property of sums. Hence there is a bijection $\prod\limits_{i\in I} \cat A(B_i,A)\iso \cat A(B,A)$, i.e. $\prod\limits_{i\in I}H_A(B_i)\iso H_A(\sum\limits_{i\in I}B_i)$. In particular $H_A(0)=1$ if $\cat A$ has an initial object $0$.
    \item Coequalizers are mapped to equalizers: Let $e$ be the equalizer of $f$ and $g$:
      \begin{align*}
        \xymatrix{
  A \ar@/^0.3pc/@[][r]^{f} \ar@/_0.3pc/@[][r]_{g} & B  \ar[r]^{e} & E
                                                       } 
      \end{align*}
      i.e. $e\of f=e\of g$. We are going to show that $H_C(e)$ is the equalizer of $H_C(f)$ and $H_C(g)$ for any $C\in\cat A$. 

      \begin{align*}
        \xymatrix{
  H_C(A) & H_C(B) \ar@/^0.3pc/@[][l]^{H_C(f)} \ar@/_0.3pc/@[][l]_{H_C(g)}  & H_C(E) \ar[l]^{H_C(e)}
                                                       } 
      \end{align*}

      Let $C$ be arbitrary. It holds $H_C(f)\of H_C(e)=H_C(g)\of H_C(e)$ since for any $h\from E\to C$, 
      \begin{align*}
        H_C(f)\of H_C(e)(h)=h\of e\of f=h\of e\of g=H_C(g)\of H_C(e)(h).
      \end{align*}
      Next, let $u\from S\to H_C(B)$ be another fork, i.e. for any $s\in S$, $H_C(f)\of u (s)=u(s)\of f=u(s)\of g=H_C(g)\of u(s)$. Hence, by the universality property, for any $s\in S$ there is a unique $\overline{u(s)}$ s.t. $u(s)=\overline{u(s)}\of e$. This immediately implies that there is a unique $\overline u\from S\to H_C(E)$ such that $u=H_C(e)\of \overline u$ (namely $\overline u(s)=\overline{u(s)}$).
    \end{itemize}
    The dual statement is that for any $A\in\cat A$, small category $\mathbf I$ and diagram $D\from \mathbf I\to\cat A$,
    $$\cat A(A,\lim_{\leftarrow I\in\mathbf I} D(I))\iso \lim_{\leftarrow I\in\mathbf I}(\cat A(A, D(I)))$$
    i.e. that the ``covariant representable functor preserves limits''.
  \item Let
    \begin{align*}
      (D(I)\xrightarrow{\alpha_I} \lim_{\to I\in\mathbf I} D(I))_{I\in\mathbf I}
    \end{align*}
    be the cocone constituting the colimit. By applying the functor $-\times F$ we get a cocone 
    \begin{align*}
      (D(I)\times F\xrightarrow{\alpha_I\times F} (\lim_{\to I\in\mathbf I} D(I))\times F)_{I\in\mathbf I}
    \end{align*} 
    since functors obviously preserve cocones. Now, due to the hint $[\cat A^\op,\Set]$ has all limits. Therefore, there is a unique natural transformation 
    \begin{align*}
      (\lim_{\to I\in\mathbf I} D(I)\times F)\xrightarrow{\overline\alpha} (\lim_{\to I\in\mathbf I} D(I))\times F
    \end{align*}
    such that the corresponding arrows commute. We are going to show that $\alpha$ is an isomorphism. By exercise Basis 3.5 it suffices to show that each component
    \begin{align*}
      (\lim_{\to I\in\mathbf I} D(I)\times F)(A)\xrightarrow{\overline\alpha_A} ((\lim_{\to I\in\mathbf I} D(I))\times F)(A)
    \end{align*}
 ($A\in\cat A$) is an isomorphism in $\Set$. 
    This can be shown with the Yoneda principle.

    Let $S\in\Set$. Then we have
    \begin{align*}
      \Set((\lim_{\to I\in\mathbf I} D(I)\times F)(A), S)&\iso \Set((\lim_{\to I\in\mathbf I} D(I)(A)\times F(A)), S)&\text{limits are computed pointwise}\\
      &\iso \lim_{\leftarrow I\in\mathbf I} \Set((D(I)(A)\times F(A)), S)&\text{\ref{ex:det}-\ref{ex:col}}\\
      &\iso \lim_{\leftarrow I\in\mathbf I} \Set(D(I)(A), S^{F(A)})&\Set \text{ is cartesian closed}\\
      &\iso \Set(\lim_{\to I\in\mathbf I} D(I)(A), S^{F(A)})&\text{\ref{ex:det}-\ref{ex:col}}\\
      &\iso \Set((\lim_{\to I\in\mathbf I} D(I)(A))\times F(A), S)&\Set \text{ is cartesian closed}\\
      &\iso \Set(((\lim_{\to I\in\mathbf I} D(I))\times F)(A), S)&\text{limits are computed pointwise}
    \end{align*}
    natural in $S$.
  \end{enumerate}
\end{answer}
\end{document}

%%% Local Variables:
%%% mode: latex
%%% TeX-master: t
%%% End:
